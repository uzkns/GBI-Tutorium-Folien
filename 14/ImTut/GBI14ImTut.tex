\documentclass[ngerman,hyperref={pdfpagelabels=false}]{beamer}

% -----------------------------------------------------------------------------

\graphicspath{{images/}}

% -----------------------------------------------------------------------------

\usetheme{KIT}

\setbeamercovered{transparent}
%\setbeamertemplate{enumerate items}[ball]

\newenvironment<>{KITtestblock}[2][]
{\begin{KITcolblock}<#1>{#2}{KITblack15}{KITblack50}}
{\end{KITcolblock}}

\usepackage[ngerman,english]{babel}
\usepackage[utf8]{inputenc}
\usepackage[TS1,T1]{fontenc}
\usepackage{array}
\usepackage{multicol}
\usepackage[absolute,overlay]{textpos}
\usepackage{beamerKITdefs}
\usepackage{amsfonts}

\newcommand{\code}[1]{\texttt{#1}}


\pdfpageattr {/Group << /S /Transparency /I true /CS /DeviceRGB>>}	%required to prevent color shifting withd transparent images


\title{Tutorium 17, \#14}
\subtitle{Max Göckel-- \textit{uzkns@student.kit.edu}}

\author[Max Göckel]{Max Göckel}
\institute{Institut für Theoretische Informatik - Grundbegriffe der Informatik}

\TitleImage[width=\titleimagewd,height=\titleimageht]{titel}

\KITinstitute{Institut f\"ur Theoretische Informatik}
\KITfaculty{Fakult\"at f\"ur Informatik}

% -----------------------------------------------------------------------------

\begin{document}
\setlength\textheight{7cm} %required for correct vertical alignment, if [t] is not used as documentclass parameter


% title frame
\begin{frame}
  \maketitle
\end{frame}


%Hoare Tripel
\begin{frame}
  \frametitle{Hoare-Kalkül: Tripel}

Ein Hoare-Tripel besteht aus zwei Zusicherungen und einem Stück Programmcode:\\
\ \\
\begin{itemize}
	\item \{P\}: Vorbedingung
	\begin{itemize}
		\item P soll wahr sein bevor der Programmcode beginnt
	\end{itemize}
	\item S: Codesegment
	\begin{itemize}
		\item Ein Stück Algorithmus - eine oder mehrere Zeilen vom Code
	\end{itemize}
	\item \{Q\}: Nachbedingung
	\begin{itemize}
		\item Ist P wahr und wurde S ausgeführt so soll nun Q wahr sein.
	\end{itemize}
\end{itemize}
\ \\
Eine Hoare-Zusicherung hat also die $\{P\} S \{Q\}$
\end{frame}


%Hoare Regeln
\begin{frame}
  \frametitle{Hoare-Kalkül: Tripel}

Für die Tripel gelten bestimmte Regeln ($ {HT}_1, {HT}_2, {HT}_3 $):\\
\ \\
\begin{itemize}
	\item[$HT_1$] Gilt $P' \rightarrow P \wedge Q \rightarrow Q'$ und ist $\{P\} S \{Q\}$ korrekt, so ist auch $\{P'\} S \{Q'\}$ korrekt
	\item[$HT_2$] $\{P\} S_1 \{Q\}$ und $\{Q\} S_2 \{R\}$ sind korrekt $\rightarrow$ $\{P\}S_1 S_2 \{R\}$ ist korrekt
	\item[$HT_3$] Ist eine eine Zuweisung $x \leftarrow \beta$, so kann man aus der Nachbed. eine Vorbed. erstellen indem man x mit $\beta$ substituiert
\end{itemize}
\end{frame}


%If Then Else
\begin{frame}
  \frametitle{Hoare-Kalkül: Verzweigungen}
In Algortihmen gibt es auch Verzweigungen der Form \\
\code{if B then S else T}. Diese können mit Hoare-Tripeln ausgewertet werden.\\
\ \\
$\{ P \}$ Verzweigung $\{ Q \}$ wahr $ \Leftrightarrow \{ P \wedge B \} S \{ Q \}$ u. $\{ P \wedge \neg B \} T \{ Q \}$ wahr\\
d.h. für eine Verzweigung sind beide Fälle wahr wenn man die Bedigung im \code{if}-Teil umdreht.\\
\end{frame}


%Hoare- beweis de rrichtigkeit
\begin{frame}
  \frametitle{Hoare-Kalkül: Beweise}
"Zeigen sie, dass das Hoare-Tripel $\{P\} S  \{Q\}$ korrekt ist". \\
\ \\
Vorgehen: 
\begin{itemize}
	\item $S$ in atomare Befehle zerlegen (dabei unten Anfang und dann hocharbeiten)
	\item Für jeden Befehl ein Tripel finden und angeben
	\item Tripel und Code verbinden
	\item Sonderfall Verzweigungen:
	\begin{itemize}
		\item Jeden Fall abarbeiten
		\item Fallbedigung $B$ herausfinden
		\item Einzelne Fälle in $\{ P \wedge B \} S \{ Q \}$ u. $\{ P \wedge \neg B \} T \{ Q \}$ einsetzen
	\end{itemize}
	\item Umformen, bis $\{P\} S  \{Q\}$ herauskommt
	\item Fertig.
\end{itemize}
\ \\
\end{frame}


%Formale Sprachen
\begin{frame}
  \frametitle{Formale Sprachen}
Gehen wir zurück zu Tutorium \#3 so sind die formalen Sprachen wie folgt definiert: \\
\ \\
\begin{KITexampleblock}{Definition}
	\begin{itemize}
		\item Eine formale Sprache F über einem Alphabet A ist eine Teilmenge der Kleenschen Hülle $A^*$
	\end{itemize}
	\end{KITexampleblock}
\ \\
\ \\
Aber was heißt das?
\end{frame}


%Formale Sprachen
\begin{frame}
  \frametitle{Die Menge $A^*$}
$A^*$ ist die Menge aller Wörter beliebiger Länge über einem Alphabet A. \\
\ \\
Ist $A = \{a, b, c\}$ so sind zum Beispiel
\begin{itemize}
	\item a
	\item abc
	\item cab
	\item abcabcabaacabcabcabcacccccabcabcabcabcabcabcbbabcbcabc
	\item $\epsilon$
\end{itemize}
alle in $A^*$
\end{frame}


%Formale Sprachen
\begin{frame}
  \frametitle{Formale Sprachen als Teilmenge von $A^*$}
Eine formale Sprache ist dabei eine Teilmenge von $A^*$, also ausgesuchte Elemente aus dieser Menge. \\
\ \\
Sei $A$ unser bekanntes Alphabet (a-zA-Z) inkl. $0,1,2,3,4,5,6,7,8,9$ und $B$ alle Sonderzeichen mit Punkt und Komma.\\
\ \\
Beispielhafte Sprachen über $A \cup B$ sind definiert als:
\begin{itemize}
	\item $ABC = \{a, b, c\}$
	\item $website = \{www. \cdot w_1 \cdot . \cdot w_2 | w_1, w_2 \in A \wedge |w_2| \leq 3\}$
	\item $DCIM = \{$IMG\_ number $ .jpeg | $number$ \in \{0,...,9\}*\}$
	\item $count_2 = \{a^n b^n | n \in \mathbb{N}\}$
\end{itemize}
\end{frame}


%Formale Sprachen
\begin{frame}
  \frametitle{Formale Sprachen: Operationen}
Auf formalen Sprachen gibt es Produkte, Potenzen und Konkatenationsabschlüsse: \\
\ \\
\begin{itemize}
	\item $L_1$, $L_2$ form. Sprachen, so ist $L_1 \cdot L_2 = \{ w_1 \cdot w_2 | w_1 \in L_1, w_2 \in L_2 \}$ also
	\item $\{a, b\} \cdot \{c, d\} = \{ ac, ad, bc, bd \}$
\end{itemize}

\begin{itemize}
	\item $L^0 = \{ \epsilon \}$ und
	\item $\forall k \in \mathbb{N}_0 : L^{k+1} = L \cdot L^k$
\end{itemize}

\begin{itemize}
	\item $L^* = \bigcup_{i \in \mathbb{N}_0} L^i$
	\item Jedes Wort aus $L$ bel. oft mit jedem anderen Wort aus $L$ kombiniert
\end{itemize}
\end{frame}


%Mastertheorem
\begin{frame}
\frametitle{Mastertheorem}
Das Mastertheorem ermöglicht Laufzeitabschätzungen für rekursive Algorithmen. \\
Die Funktionen der Algorithmen müssen folgende Form besitzen, damit das Mastertheorem angewendet werden kann: \\
$T(n) = a \cdot T(\frac{n}{b}) + f(n)$.
\end{frame}


\begin{frame}
\frametitle{Mastertheorem}
$T(n) = a \cdot T(\frac{n}{b}) + f(n)$. \\
\ \\
\begin{itemize}
\item $a$: Anzahl der Teilprobleme
\item $\frac{n}{b}$: Größe der Teilprobleme
\item $f(n)$: Von $T(n)$ unabhängige Funktion zur Kombination der Teilprobleme
\end{itemize}

Bsp.: $T_1(n) = a \cdot T_1(\frac{n}{10}) + 3n^2$
\end{frame}


%Mastertheorem Fälle
\begin{frame}
\frametitle{Mastertheorem: Fälle}
Bei der Laufzeit der Form $T(n) = a \cdot T(\frac{n}{b}) + f(n)$ gibt es 3 Fälle:\\
\ \\
\begin{enumerate}
\item Ist $f(n) \in O( n^{{log}_{b}a - \epsilon} )$ für ein $\epsilon > 0 \Rightarrow T(n) \in \Theta( n^{{log}_{b} a} )$
\item Ist $f(n) \in \Theta( n^{{log}_b a} ) \Rightarrow T(n) \in \Theta( n^{{log}_b a} \cdot {log}(n) )$
\item Ist $f(n) \in \Omega( n^{{log}_b a+\epsilon} )$ für ein $\epsilon > 0$ und $\exists d \in (0,1)$, sodass für ein großes $n$ gilt: $a \cdot f(\frac{n}{b}) \leq d \cdot f(n) \Rightarrow T(n) \in \Theta( f(n) )$
\end{enumerate}
\end{frame}


%LÖSUNG
\begin{frame}
\frametitle{Mastertheorem: Beispiel}
$4 \cdot T(\frac{n}{4} + n^2$ \\
\ \\
${log}_b a = 1$, \\
$f(n) = n^2 \in \Omega( n^{{log}_b a+\epsilon} ) = \Omega( n^{1+\epsilon} )$, \\
Für $\epsilon = \frac{1}{2}$. Es soll gelten $a f(\frac{n}{b}) \leq d f(n), d \in (0,1)$.\\
\ \\
$a f(\frac{n}{b}) = 4 f(\frac{n}{4}) = \frac{n^2}{4} \leq d n^2 = d f(n)$ mit $d = \frac{1}{4}$. \\
$\Rightarrow T(n) \in \Theta(n^2)$

\end{frame}


%Vorgehen
\begin{frame}
\frametitle{Vollständige Induktion}
Drei Schritte:\\
\ \\
\begin{enumerate}
	\item Induktionsanfang (IA): Den kleinsten Wert nehmen und die Behauptung für diesen zeigen. Manchmal noch die Behauptung für den ersten Schritt zeigen.
	\item Induktionsvoraussetzung: Die Behauptung <Behauptung> gilt für ein beliebiges aber festes $n \in \mathbb{N}_+$ (oder worüber man die Induktion anwendet).
	\item Induktionsschritt (IS): Für alle $n$ ein $(n+1)$ einsetzen und so lange Umformen bis der Ursprungsterm (mit dem $n$) wieder auftaucht.
\end{enumerate}
\end{frame}



%Vorgehen
\begin{frame}
\frametitle{Induktion}
Bei der Induktion über kontextfreie Grammatiken ist oft $n$ die Länge des Wortes oder die Anzahl an Ableitungsschritten. \\
\ \\
Hierbei dann davon ausgehen das an einer Stelle ein Nichtterminalsymbol steht und jeden möglichen Ableitungsschritt erklären um die Behauptung für $(n+1)$ zu zeigen.
\end{frame}


%Speicher
\begin{frame}
  \frametitle{Speicher}
\begin{KITinfoblock}{Definition}
	\begin{itemize}
		\item $memwrite: Val^{Adr} x Adr x Val \rightarrow Val^{Adr}.$\\
		\begin{itemize}
			\item $(m, a, v) \mapsto m'$
		\end{itemize}
	\end{itemize}
\end{KITinfoblock}
\ \\
\ \\
\begin{KITinfoblock}{Definition}
	\begin{itemize}
		\item $memread: Val^{Adr} x Adr \rightarrow Val$\\
		\begin{itemize}
			\item $(m, a) \mapsto m(a)$
		\end{itemize}
	\end{itemize}
\end{KITinfoblock}
\ \\
\ \\
Die Operationen können auch verkettet werden. an jedem $Val^{Adr}$ kann $memwrite$ einsetzt werden um die Tabelle vor dem Lesen zu verändern.
\end{frame}


%MIMA Aufbau%
\begin{frame}
\frametitle{Aufbau der MIMA}
\begin{itemize}
	\item Steuerwerk
	\begin{itemize}
		\item Holt Befehle aus dem Speicher, decodiert sie und steuert die Ausführung
	\end{itemize}
	\item Rechenwerk
	\begin{itemize}
		\item Führt arithmetische/logische Operationen aus
	\end{itemize}
	\item Speicher
	\begin{itemize}
		\item Speichert Daten an Adressen, besteht aus Programm- und Datenspeicher
		\item Adressen 20Bit, Werte 24Bit 
	\end{itemize}
	\item Register
	\begin{itemize}
		\item Speichern je ein Wort oder eine Adresse (20/24Bit)
	\end{itemize}
	\item Bus
	\begin{itemize}
		\item Verbindet die Komponenten
	\end{itemize}
\end{itemize}
\end{frame}


%GBI Themen
\begin{frame}
\frametitle{Klasusurthemen}
Alle Themen aus GBI: \\
\begin{columns}
	\begin{column}{0.5\textwidth}
		\begin{itemize}
			\item Mengen
			\item Relationen
			\item Wörter
			\item Formale Sprachen
			\item Kontextfreie Grammatiken
			\item Induktion
			\item Aussagenlogik
			\item Prädikatenlogik
			\item Huffman
		\end{itemize}
	\end{column}
	\begin{column}{0.5\textwidth}  %%<--- here
		\begin{center}
			\begin{itemize}
				\item Speicher
				\item MIMA
				\item Hoare-Kalkül
				\item Graphen
				\item O-Kalkül/Mastertheorem
				\item Automaten
				\item RegEx
				\item Turingmaschinen
			\end{itemize}
		\end{center}
	\end{column}
\end{columns}
\end{frame}

\end{document}
