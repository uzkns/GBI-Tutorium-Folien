\documentclass[ngerman,hyperref={pdfpagelabels=false}]{beamer}

% -----------------------------------------------------------------------------

\graphicspath{{images/}}

% -----------------------------------------------------------------------------

\usetheme{KIT}

\setbeamercovered{transparent}
%\setbeamertemplate{enumerate items}[ball]

\newenvironment<>{KITtestblock}[2][]
{\begin{KITcolblock}<#1>{#2}{KITblack15}{KITblack50}}
{\end{KITcolblock}}

\usepackage[ngerman,english]{babel}
\usepackage[utf8]{inputenc}
\usepackage[TS1,T1]{fontenc}
\usepackage{array}
\usepackage{multicol}
\usepackage[absolute,overlay]{textpos}
\usepackage{beamerKITdefs}
\usepackage{amsfonts}


\pdfpageattr {/Group << /S /Transparency /I true /CS /DeviceRGB>>}	%required to prevent color shifting withd transparent images


\title{Tutorium 17, \#5}
\subtitle{Max Göckel-- \textit{uzkns@student.kit.edu}}

\author[Max Göckel]{Max Göckel}
\institute{Institut für Theoretische Informatik - Grundbegriffe der Informatik}

\TitleImage[width=\titleimagewd,height=\titleimageht]{titel}

\KITinstitute{Institut f\"ur Theoretische Informatik}
\KITfaculty{Fakult\"at f\"ur Informatik}

% -----------------------------------------------------------------------------

\begin{document}
\setlength\textheight{7cm} %required for correct vertical alignment, if [t] is not used as documentclass parameter


% title frame
\begin{frame}
  \maketitle
\end{frame}


%Speicher
\begin{frame}
\frametitle{Speicher}
Man kann sich den Speicher als eine Art Tabelle vorstellen. Die linke Spalte sind die Adressen, rechte Spalte die Daten an der jeweiligen Adresse.\\
Bei einer Tabelle wird immer nur ein Zustand gespeichert, nicht eine "Referenz" auf die Tabelle.
\ \\
Bsp: Speicher $m_0$
\begin{tabular}{c|c}
Stelle & Wert \\ \hline
00 & 010 \\
01 & 000 \\
10 & 111 \\
11 & 100 \\
\end{tabular}
\end{frame}


%Schreiben in den Speicher
\begin{frame}
\frametitle{Speicheroperationen: Schreiben}
Schreiben in den Speicher mit:\\

\begin{KITinfoblock}{Definition}
	\begin{itemize}
		\item $memwrite: Val^{Adr} x Adr x Val \rightarrow Val^{Adr}.$\\
		\begin{itemize}
			\item $(m, a, v) \mapsto m'$
		\end{itemize}
	\end{itemize}
	\end{KITinfoblock}
\ \\
\ \\
Dies schreibt den Wert \textbf{v} an Adresse \textbf{a} von Speichertabellenzustand \textbf{m} und gibt den veränderten Zustand \textbf{m'} zurück.
\end{frame}


%Lesen aus dem Speicher
\begin{frame}
\frametitle{Speicheroperationen: Lesen}
Lesen aus dem Speicher mit:\\

\begin{KITinfoblock}{Definition}
	\begin{itemize}
		\item $memread: Val^{Adr} x Adr \rightarrow Val$\\
		\begin{itemize}
			\item $(m, a) \mapsto m(a)$
		\end{itemize}
	\end{itemize}
	\end{KITinfoblock}
\ \\
\ \\
Dies liest aus der Adresse \textbf{a} von Zustand \textbf{m} aus und gibt diesen Wert zurück.
\end{frame}


%Beispiel
\begin{frame}
\frametitle{Speicheroperationen: Verkettung}
Da die Operationen einen Rückgabewert haben, können wir sie Anstelle von konkreten Werten in anderen Operationen einsetzen.\\
\ \\
Beispiel: Wert aus Stelle 00 auslesen und in 01 einsetzen.\\
\ \\
\ \\
$m_1$:
\begin{tabular}{c|c}
Stelle & Wert \\ \hline
00 & 010 \\
01 & 000 \\
10 & 111 \\
11 & 100 \\
\end{tabular}
\end{frame}


%Aufgabe
\begin{frame}
\frametitle{Speicher: Aufgabe}
Wie sieht $m_3$ nach der folgenden verketteten Operation auf Tabelle $m_2$ aus?\\
\ \\
\begin{itemize}
	\item $m_3 = memwrite(m_2, 11, memread(memwrite(m_2, 10, 000), 00))$
\end{itemize}
\ \\
\ \\
$m_2$:
\begin{tabular}{c|c}
Stelle & Wert \\ \hline
00 & 010 \\
01 & 000 \\
10 & 111 \\
11 & 100 \\
\end{tabular}
\end{frame}


%LÖSUNG%
\begin{frame}
\frametitle{Speicher: Lösung}
Wie sieht $m_3$ nach der folgenden verketteten Operation auf Tabelle $m_2$ aus?\\
\ \\
\begin{itemize}
	\item $m_3 = memwrite(m_2, 11, memread(memwrite(m_2, 10, 000), 00))$
\end{itemize}
\ \\
\ \\
$m_3$:
\begin{tabular}{c|c}
Stelle & Wert \\ \hline
00 & 010 \\
01 & 000 \\
10 & 111 \\
11 & 010 \\
\end{tabular}
\end{frame}


%MIMA Aufbau%
\begin{frame}
\frametitle{Die MIMA: Aufbau}
\begin{itemize}
\item Steuerwerk
\begin{itemize}
	\item Holt Befehle aus dem Speicher, decodiert sie und steuert die Ausführung
\end{itemize}
\item Rechenwerk
\begin{itemize}
	\item Führt arithmetische/logische Operationen aus
\end{itemize}
\item Speicher
\begin{itemize}
	\item Speichert Daten an Adressen, besteht aus Programm- und Datenspeicher
	\item Adressen 20Bit, Werte 24Bit 
\end{itemize}
\item Register
\begin{itemize}
	\item Speichern je ein Wort oder eine Adresse (20/24Bit)
\end{itemize}
\item Bus
\begin{itemize}
	\item Verbindet die Komponenten
\end{itemize}
\end{itemize}
\end{frame}


%ALU
\begin{frame}
\frametitle{Die MIMA: ALU}
Führt die vom Steuerwerk decodierten Befehle aus (logisch oder arithmetisch).\\
\ \\
\begin{itemize}
	\item X, Y: Eingabewerte
	\item Z: Ausgabe der verarbeiteten Werte
\end{itemize}
\end{frame}


%ALU Operationen
\begin{frame}
\frametitle{Die MIMA: ALU-Operationen}
Bitweise Operationen der ALU:.\\
\ \\
\begin{tabular}{c|c}
Name & Operation in Worten \\ \hline
ADD & Zahl 1 + Zahl 2 \\
AND & 1 $\Leftrightarrow$ beide Eingabwerte 1 sind (logische $\wedge$)\\
OR & 1, wenn min. 1 wert 1 ist (logisches $\vee$)\\
NOT & Jeder wert wird invertiert (Negierung) \\
XOR & 1 wenn genau ein Wert 1 ist (exclusive or)\\
\end{tabular}
\end{frame}


%Register%
\begin{frame}
\frametitle{Die MIMA: Register}
\begin{itemize}
	\item Akku: Akkumulator (Zwischenspeicher)
	\item X, Y: ALU-Operanden
	\item Z: ALU-Ergebnis
	\item EINS: Die Konstante 1
	\item IAR: Instruktionsadressreg., Speichert den Speicherwert des nächsten Befehls
	\item IR: Instruktionsreg., Speichert die derzeitige Instruktion
	\item SAR: Speicheradressreg., (write-only) ruft im Speicher seinen wert auf
	\item SDR: Speicherdatenreg., Wert$\rightarrow$Speicher oder Speicher$\rightarrow$Wert
\end{itemize}

\end{frame}

%Befehle%
\begin{frame}
\frametitle{Die MIMA: Befehle}
\begin{itemize}
	\item Maschinenbefehle/Assemblersprache: Rechen- und Sprungbefehle
	\item Mikrobefehle: Bitfolgen, die an das Steuerwerk gesendet werden
\end{itemize}
\ \\
Die Mikrobefehle werden in 7+ Takten abgearbeitet. Wie lange ein Takt (bzw. wann der nächste Takt beginnt) steuert ein Taktquarz im Steuerwerk.
\end{frame}


%Befehle in ASM%
\begin{frame}
\frametitle{Die MIMA: Befehle}
Auf einer Extra-PDF im ILIAS.\\
\ \\
\begin{itemize}
	\item Die Befehle werden aufsteigend durchnummeriert und dann abgearbeitet
	\item Alle Werte immer in binär aufschreiben!
	\item Es ist nicht möglich mehr als 20Bit lange Worte zu laden und es ist auch nicht möglich negative Zahlen zu laden
	\item Kommentare werden im Format \emph{;<Kommentar>} geschrieben
	\item Jedes Assemblerprogramm endet mit einem \textit{HALT} (Punktabzug!)
\end{itemize}
\end{frame}


%Aufgabe%
\begin{frame}
\frametitle{Die MIMA: Aufgabe}
Was tut das folgende Programm?\\
\ \\
\qquad LDC 1 \\
\qquad STV 011 \\
\qquad LDV 010 \\
\qquad NOT \\
\qquad ADD 011 \\
\qquad ADD 001 \\
\qquad JMN m1 \\
\qquad LDC 1 \\
\qquad STV 111 \textit{;result}\\
\qquad JMP m2 \\
m1: LDC 0 \\
\qquad STV 111 \textit{;result}\\
m2: HALT
\end{frame}


%Aufgabe%
\begin{frame}
\frametitle{Die MIMA: Aufgabe}
Es zieht den Wert an Stelle $010$ von dem An Wert $001$ ab, falls das Ergebnis positiv ist, speichert es $1$ in $111$, sonst $0$: \\
Anders: \\
\ \\
$pos(x,y) = \left\{
\begin{array}{ll}
1 & \textrm{wenn } x-y \geq 0 \\
0 & \, \textrm{wenn } x-y < 0 \\
\end{array}
\right. $
\end{frame}


%Aufgabe%
\begin{frame}
\frametitle{Die MIMA: Aufgabe}
Schreibe ein Programm, dass den Wert an Adresse 001 verdreifacht.
\end{frame}


%LÖSUNG%
\begin{frame}
\frametitle{Die MIMA: Lösung}
Schreibe ein Programm, dass den Wert an Adresse 001 verdreifacht.\\
\ \\
LDV 001 \\
ADD 001 \\
ADD 001 \\
STV 001 \\
HALT.
\end{frame}



%Aufgabe%
\begin{frame}
\frametitle{Die MIMA: Aufgabe}
Schreibe ein Programm dass ein ganzzahliges $x$ aus der Stelle 111 lädt, $-|x|$ bildet und das Ergbnis wieder in 11 speichert.\\
\end{frame}


%LÖSUNG%
\begin{frame}
\frametitle{Die MIMA: Lösung}
Schreibe ein Programm dass ein ganzzahliges $x$ aus der Stelle 111 lädt, $-|x|$ bildet und das Ergbnis wieder in 11 speichert.\\
\ \\
\qquad LDV 001 \\
\qquad JMN m1 \\
\qquad NOT \\
\qquad ADD EINS \\
\qquad STV 001 \\
m1: HALT \\

\end{frame}


%Aufgabe%
\begin{frame}
\frametitle{Die MIMA: Aufgabe}
Realisiert die Funktion: \\
\ \\
$max(a, b) = \left\{
\begin{array}{ll}
b & b \geq a \\
a & \, b < a \\
\end{array}
\right. $ \\
\ \\
\ \\
Wobei $a$ in 001 steht, $b$ in 010 und das Ergebnis in beide Speicherplätze (001,010) geschrieben werden soll.
\end{frame}


%LÖSUNG%
\begin{frame}
\frametitle{Die MIMA: Lösung}
\qquad LDV 001 \\
\qquad NOT \\
\qquad ADD EINS \\
\qquad ADD 010 \\
\qquad JMN m1 \\
\qquad LDV 010 \\
\qquad STV 001 \\
\qquad JMP m2 \\
m1: LDV 001 \\
\qquad STV 010 \\
m2: HALT. \\
\end{frame}





\end{document}
