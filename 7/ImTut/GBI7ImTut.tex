\documentclass[ngerman,hyperref={pdfpagelabels=false}]{beamer}

% -----------------------------------------------------------------------------

\graphicspath{{images/}}

% -----------------------------------------------------------------------------

\usetheme{KIT}

\setbeamercovered{transparent}
%\setbeamertemplate{enumerate items}[ball]

\newenvironment<>{KITtestblock}[2][]
{\begin{KITcolblock}<#1>{#2}{KITblack15}{KITblack50}}
{\end{KITcolblock}}

\usepackage[ngerman,english]{babel}
\usepackage[utf8]{inputenc}
\usepackage[TS1,T1]{fontenc}
\usepackage{array}
\usepackage{multicol}
\usepackage[absolute,overlay]{textpos}
\usepackage{beamerKITdefs}
\usepackage{amsfonts}


\pdfpageattr {/Group << /S /Transparency /I true /CS /DeviceRGB>>}	%required to prevent color shifting withd transparent images


\title{Tutorium 17, \#7}
\subtitle{Max Göckel-- \textit{uzkns@student.kit.edu}}

\author[Max Göckel]{Max Göckel}
\institute{Institut für Theoretische Informatik - Grundbegriffe der Informatik}

\TitleImage[width=\titleimagewd,height=\titleimageht]{titel}

\KITinstitute{Institut f\"ur Theoretische Informatik}
\KITfaculty{Fakult\"at f\"ur Informatik}

% -----------------------------------------------------------------------------

\begin{document}
\setlength\textheight{7cm} %required for correct vertical alignment, if [t] is not used as documentclass parameter


% title frame
\begin{frame}
  \maketitle
\end{frame}


%Grammatiken
\begin{frame}
\frametitle{Kontextfreie Grammatiken}
Kontextfreie Grammatiken sind ein Viertupel G = (N, T, S, P) mit:\\
\ \\
\begin{itemize}
	\item N: Das Alphabet der Nonterminalsymbole
	\item T: Das Alphabet der Terminalsymbole
	\item S: dem Startsymbol (mit $S \in N$)
	\item P: einer (endlichen) Menge an Produktionen
\end{itemize}
\end{frame}


%Beispiel
\begin{frame}
\frametitle{Kontextfreie Grammatiken: Beispiel}
Beispiel-Viertupel:\\
\ \\
\begin{itemize}
	\item $G = ( \{X, Y\}, \{a, b, c\}, X, P_1 )$
	\item $P_1 = ( X \rightarrow aX | bY, Y \rightarrow c )$
\end{itemize}
\end{frame}


%Beispiel Lösung
\begin{frame}
\frametitle{Kontextfreie Grammatiken: Beispiel}
Beispiel-Viertupel:\\
\ \\
\begin{itemize}
	\item $G = ( \{X, Y\}, \{a, b, c\}, X, P_1 )$
	\item $P_1 = ( X \rightarrow aX | bY, Y \rightarrow c )$
\end{itemize}
\ \\
Eine mögliche Ableitung ist dann:\\
$X \Rightarrow aX \Rightarrow aaX \Rightarrow aabY \Rightarrow aabc$ \\
oder:\\
$X \Rightarrow bY \Rightarrow bc$
\end{frame}


%Besondere Pfeile
\begin{frame}
\frametitle{Kontextfreie Grammatiken: Pfeile}
Einzelne Ableitung: \\
\begin{itemize}
	\item $ {\Rightarrow} $
\end{itemize}
Ableitung mit n Schritten: \\
\begin{itemize}
	\item $ {\Rightarrow}^n $
\end{itemize}
Ableitung mit bel. Schritten: \\
\begin{itemize}
	\item $ {\Rightarrow}^* $
\end{itemize}
\end{frame}


%Sprache einer Grammatik
\begin{frame}
\frametitle{Kontextfreie Grammatiken: Sprachen}
Die Spache L(G) der Grammatik G ist die Menge aller Wörter die mit G abgeleitet werden können und die nur Terminalsybole enthalten.\\
\end{frame}


%Sprache einer Grammatik
\begin{frame}
\frametitle{Kontextfreie Grammatiken: Sprachen}
Die Spache L(G) der Grammatik G ist die Menge aller Wörter die mit G abgeleitet werden können und die nur Terminalsybole enthalten.\\
\ \\
Im vorigen Beispiel ist $L(G) = \{ \{a\}^+ \{b\}^+ c\}$
\end{frame}


%Aufgabe
\begin{frame}
\frametitle{Kontextfreie Grammatiken: Aufgabe}
Gibt es eine kontextfreie Grammatik mit $L(G) = \emptyset$? Wie sieht sie aus?
\end{frame}


%LÖSUNG
\begin{frame}
\frametitle{Kontextfreie Grammatiken: Lösung}
Gibt es eine kontextfreie Grammatik mit $L(G) = \emptyset$? Wie sieht sie aus?\\
\ \\
Ja, und zwar $G_{ \emptyset } = ( \{X\}, \{q\}, X, \{X \rightarrow X\})$.\\
$G_{\emptyset}$ produziert nie ein Wort ohne Nonterminalsymbole.
\end{frame}


%Ableitungsbäume
\begin{frame}
\frametitle{Kontextfreie Grammatiken: Ableitungsbäume}
Zu einer Grammatik G kann man die Ableitungen zu einem Wort w $\in$ L(G) auch als einen Ableitungsbaum schreiben.\\
Dieser stellt die Schritte grafisch dar und hilft bei der Darstellung einer Grammatik.
\end{frame}


%Aufgabe
\begin{frame}
\frametitle{Kontextfreie Grammatiken: Aufgaben}
Sei $G_1 = \{ \{ X, Y \}, \{ a, b, c \}, X, P_2 \}$\\
$P_2 = \{ X \rightarrow aXa | bXb | Y, Y \rightarrow cY | \epsilon \}$\\
\ \\
Ist $abccba \in L(G)$? Zeige es mit den Ableitungen und dem Ableitungsbaum. \\
\ \\
\ \\
Sei $G_2 = \{ \{ S, U, X, Q \}, \{a\}, S, P_3 \}$\\
$P_3 = \{$ \\
	$S \rightarrow aU | aXa | Qaa, $\\
	$U \rightarrow aaU | \epsilon,$ \\
	$X \rightarrow Qaaa | a, $\\
	$Q \rightarrow aXa | a \}$\\
\ \\
Leitet $a^7$ ab.
\end{frame}


%LÖSUNG
\begin{frame}
\frametitle{Kontextfreie Grammatiken: Lösung}
Sei $G = \{ \{ X, Y \}, \{ a, b, c \}, X, P_2 \}$\\
$P_2 = \{ X \rightarrow aXa | bXb | Y, Y \rightarrow cY | c \}$\\
Ist $abccba \in L(G)$? Zeige es mit den Ableitungen und dem Ableitungsbaum.\\
\ \\
abcba ist in L(G), da es durch $X \Rightarrow aXa \Rightarrow abXba \Rightarrow abYba \Rightarrow abcYba \Rightarrow abccba$ abgeleitet werden kann.\\
\ \\
Sei $G_2 = \{ \{ S, U, X, Q \}, \{a\}, S, P_3 \}$\\
$P_3 = \{$ 
	$S \rightarrow aU | aXa | Qaa, $
	$U \rightarrow aaU | \epsilon,$ 
	$X \rightarrow Qaaa | a, $
	$Q \rightarrow aXa | a \}$
Leitet $a^7$ ab.\\
\ \\
$S \Rightarrow aU \Rightarrow aaaU \Rightarrow aaaaaU \Rightarrow aaaaaaaU \Rightarrow aaaaaaa$
\end{frame}


%Relationen Teil 2
\begin{frame}
\frametitle{Relationen: Rückblick}
Wir kennen bereits:\\
\ \\
Relation R $\subseteq$ A $\times$ B, also enthält R Tupel aus der Menge A $\times$ B.\\
Im Fall  R $\subseteq$ A $\times$ A heißt R \emph{"Relation auf A"}.\\
\ \\
Relationen haben 4 Eigenschaften:\\
\begin{itemize}
	\item Linkstotal
	\item Rechtstotal
	\item Linkseindeutig
	\item Rechtseindeutig
\end{itemize}
\end{frame}


%Identität
\begin{frame}
\frametitle{Relationen: Identität}
Die Identität über einer Menge M ist die Relation $I_M$ oder auch die Identität.\\
Die Identität ist die Abbildung $f(x) = x$ als Relation formuliert, sie ändert nichts an der Menge, dazu ist sie das neutrale element der Verkettung ($\circ$)\\
\ \\
Formal: $I_M = \{(x, x) | x \in M\}$
\end{frame}


%Relationen Definitionen
\begin{frame}
\frametitle{Relationen: Definitionen}
Seien $R \subseteq M_1 \times M_2, S \subseteq M_2 \times M_3$ zwei Relationen.\\
\ \\
Produkt von Relationen:\\
$S \circ R = \{ (x, z)  \in M_1 \times M_3 | \exists y \in M_2: (x,y) \in R \wedge (y, z) \in S \}$\\
\ \\
Potenz einer Relation:\\
$R^n = R \circ R \circ  ... \circ R$, n $\in \mathbb{N}_0$-mal \\
$R^0 = I_M$, die Identität\\
$R^*$ ist die reflexiv-transitive Hülle von R
\end{frame}


%Relationen Eigenschaften
\begin{frame}
\frametitle{Relationen: Eigenschaften}
Zu den schon bekannten Eigenschaften gibt es noch 3 neue Eigenschaften von Relationen:\\
\ \\
\begin{itemize}
	\item reflexiv: R ist reflexiv $\Leftrightarrow I_M \subseteq R$
	\item symmetrisch: R ist symm. $\Leftrightarrow \forall (x,y) \in R: (y,x) \in R$
	\item transitiv: R ist transitiv $\Leftrightarrow \forall (x,y) \in R \wedge (y,z) \in R \rightarrow (x,z) \in R$
\end{itemize}
\end{frame}


%Relationen Aufgabe
\begin{frame}
\frametitle{Relationen: Aufgabe}
Welche Eigenschaften haben die Relationen?\\
\ \\
\begin{tabular}{c|c|c|c|c}
 & $x = y$ & $ x \le y$ & $x < y$ & $x \neq y$\\ \hline
reflexiv & & & & \\
symmetrisch & & & & \\
transitiv & & & & \\
\end{tabular}
\end{frame}


%LÖSUNG
%Relationen Aufgabe
\begin{frame}
\frametitle{Relationen: Lösung}
Welche Eigenschaften haben die Relationen?\\
\ \\
\begin{tabular}{c|c|c|c|c}
 & $x = y$ & $ x \le y$ & $x < y$ & $x \neq y$\\ \hline
reflexiv & \checkmark & \checkmark & x & x \\
symmetrisch & \checkmark & x & x & \checkmark \\
transitiv & \checkmark & \checkmark & \checkmark & x \\
\end{tabular}
\end{frame}


\end{document}
