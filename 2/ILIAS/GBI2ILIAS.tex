\documentclass[ngerman,hyperref={pdfpagelabels=false}]{beamer}

% -----------------------------------------------------------------------------

\graphicspath{{images/}}

% -----------------------------------------------------------------------------

\usetheme{KIT}

\setbeamercovered{transparent}
%\setbeamertemplate{enumerate items}[ball]

\newenvironment<>{KITtestblock}[2][]
{\begin{KITcolblock}<#1>{#2}{KITblack15}{KITblack50}}
{\end{KITcolblock}}

\usepackage[ngerman,english]{babel}
\usepackage[utf8]{inputenc}
\usepackage[TS1,T1]{fontenc}
\usepackage{array}
\usepackage{multicol}
\usepackage[absolute,overlay]{textpos}
\usepackage{beamerKITdefs}
\usepackage{amsfonts}


\pdfpageattr {/Group << /S /Transparency /I true /CS /DeviceRGB>>}	%required to prevent color shifting withd transparent images


\title{Tutorium 42, \#2}
\subtitle{Max Göckel-- \textit{uzkns@student.kit.edu}}

\author[Max Göckel]{Max Göckel}
\institute{Institut für Theoretische Informatik - Grundbegriffe der Informatik}

\TitleImage[width=\titleimagewd,height=\titleimageht]{titel}

\KITinstitute{Institut f\"ur Theoretische Informatik}
\KITfaculty{Fakult\"at f\"ur Informatik}

% -----------------------------------------------------------------------------

\begin{document}
\setlength\textheight{7cm} %required for correct vertical alignment, if [t] is not used as documentclass parameter


% title frame
\begin{frame}
  \maketitle
\end{frame}

%Alphabete
\begin{frame}
  \frametitle{Rückblick: Alphabete}


\begin{KITexampleblock}{Definition}
	\begin{itemize}
		\item Ein Alphabet ist eine \emph{endliche, nichtleere} Menge aus Zeichen / Symbolen. Was dabei ein Zeichen ist, ist nicht eingeschränkt.
	\end{itemize}
	\end{KITexampleblock}\\
\ \\
Beipielalphabete:
\begin{enumerate}
\item \{H, a, n, d, y\}
\item \{Handy\}
\item \{Ha, ndy\}
\end{enumerate}
Können alle "Handy" erstellen/schreiben
\end{frame}

%Worte
\begin{frame}
  \frametitle{Worte}
\begin{KITexampleblock}{Definition}
	\begin{itemize}
		\item Ein Wort w aus einem Alphabet A ist eine Folge von Zeichen aus A 
		
	\end{itemize}
	\end{KITexampleblock}\\
\ \\
Beipielworte aus A = \{H, a, n, d, y, -, 1, 2, 3, 4 ,5, 6, 7, 8, 9, 0\}
\begin{enumerate}
\item Handy
\item H1a2n3d4y5
\item ---aa-----HH1-------
\item 017341856397
\end{enumerate}

\end{frame}

%Folgen
\begin{frame}
  \frametitle{Folgen}


\begin{KITexampleblock}{Definition}
	\begin{itemize}
		\item Eine Folge ist eine Auflistung von Objekten, welche fortlaufend nummeriert sind.
	\end{itemize}
	\end{KITexampleblock}\\
\ \\
Wofür brauchen wir Folgen?\\
\ \\
\begin{tabular}{ccccccccccccc}
1 & 2 & 3 & 4 & 5 & 6 & 7 & 8 & 9 & 10 & 11 & 12 & 13\\
G & r & u & n & d & b & e & g & r & i & f & f & e\\
\end{tabular}\\
\ \\
\begin{itemize}
\item 13tes Zeichen aus dem Wort? \emph{e}.
\item Länge des Wortes? \emph{13}.
\end{itemize}
\end{frame}

%Wörter als Abbildung
\begin{frame}
  \frametitle{Worte als Abbildungen}


\begin{KITexampleblock}{Definition}
	\begin{itemize}
		\item Ein Wort ist eine surjektive Abb. $w: \mathbb{Z}_n \rightarrow B$ mit $B \subseteq A$
	\end{itemize}
	\end{KITexampleblock}\\
\ \\
formal: w = Handy\\
$w: \mathbb{Z}_5 \rightarrow \{H, a, n, d, y\}$\\
mit w(0) = H, w(1) = a, w(2) = n, w(3) = d, w(4) = y
\end{frame}

%Leerzeichen
\begin{frame}
  \frametitle{Leerzeichen}

\begin{KITalertblock}{Achtung}
	\begin{itemize}
		\item Ein Leerzeichen ist auch nur wieder ein Symbol. es trennt Wörter nach der Definition nicht
	\end{itemize}
	\end{KITalertblock}\\
\ \\
Beipielwort aus A = \{H, a, l, o, W, e, t, \} ist w = Hallo Welt\\
\ \\
\begin{itemize}
\item Eine Folge von Zeichen
\item Ein Wort, nicht zwei (auch wenn durch Leerzeichen getrennt)
\item Leerzeichen manchmal auch ␣ geschrieben
\end{itemize}
\end{frame}


%Leeres Wort
\begin{frame}
  \frametitle{Leeres Wort}

\begin{KITexampleblock}{Definition}
	\begin{itemize}
		\item Das leere Wort ist die Abbildung $\epsilon : \mathbb{Z}_0 \rightarrow \{ \}$
	\end{itemize}
	\end{KITexampleblock}\\
\ \\
Das leere Wort hat Länge $| \epsilon |$ = 0, da es aus 0 Zeichen besteht
\end{frame}

%Konkatenation
\begin{frame}
  \frametitle{Konkatenation}

\begin{KITexampleblock}{Definition}
	\begin{itemize}
\item $|w_1| = m$ und $|w_2| = n$
	\item $w_1 \cdot w_2 : \mathbb{Z}_{m+n} \rightarrow A_1 \cup A_2$. $ 
    	 i \mapsto \left\{\begin{array}{ll}
				w_1 (i), & 0 \leq i < m \\
         				w_2 (i-m), & m \leq i < m + n \\
			\end{array} \right.$
	\end{itemize}
\end{KITexampleblock}\\
\ \\
\begin{itemize}
\item Hintereinanderschreiben von 2 Worten
\item Getrennt durch einen $\cdot$, kann auch weggelassen werden
\item Zuerst die m Buchstaben des ersten Wortes, dann die n Buchstaben des zweiten Wortes
\item leeres Wort $\epsilon$ ist neutrales Element der Konkatenation ($w \cdot \epsilon = \epsilon \cdot w = w$)
\item Konkatenation ist nicht kommutativ, aber assoziativ
\end{itemize}
\end{frame}

%Potenzen
\begin{frame}
  \frametitle{Potenzen}

\begin{KITexampleblock}{Definition}
	\begin{itemize}
		\item A* ist die Menge aller Wörter über dem Alphabet A
	\end{itemize}
\end{KITexampleblock}\\
\ \\
\begin{KITexampleblock}{Definition}
	\begin{itemize}
		\item $A^n$ ist die Menge aller Wörter der Länge n über dem Alphabet A
	\end{itemize}
\end{KITexampleblock}\\
\ \\
\begin{KITexampleblock}{Definition}
	\begin{itemize}
		\item $w^n$ ist die n-fache Aneinanderreihung des Wortes w mit $w^0 = \epsilon$
	\end{itemize}
\end{KITexampleblock}
\end{frame}

\begin{frame}
\frametitle{Aussagen}
Was sind sind sinnvolle Aussagen?\\\
\begin{itemize}
	\item "Dieses Tutorium findet wöchentlich statt." \textbf{Wahr}
	\item $f: \mathbb{N}_+ \rightarrow \mathbb{R}, x \mapsto 2x$ ist surjektiv. \textbf{Falsch}
	\item Wenn es regnet wird die Straße nass. \textbf{Wahr}
	\item Grün ist toll \emph{Keine Ahnung}
	\item Regnet es? \emph{Wo?}
\end{itemize}
\end{frame}


%Aussagen
\begin{frame}
\frametitle{Aussagen: Grundlagen}
Aussagen sind entweder objektiv wahr oder falsch, nichts dazwischen. Man kann sie mit Ja oder Nein beantworten.
\end{frame}


%Verknüpfung
\begin{frame}
\frametitle{Aussagen: Verknüpfung}
Wir können die Grundaussagen miteinander verknüpfen. Seien A, B zwei Aussagen so gibt es folgende Verknüpfungen:\\
\ \\
\begin{tabular}{cc|ccccc}
A & B & $\neg A$ & $A \vee B$ & $A \wedge B$ & $A \rightarrow B$ & $A \leftrightarrow B$\\ \hline
f  & f  &   w & f  & f  & w  & w\\
f  & w &   w & w & f  & w & f\\
w & f  &   f  & w & f  & f   & f\\
w & w &   f  & w & w & w & w\\
\end{tabular}\\
\end{frame}


%Verknüpfung II
\begin{frame}
\frametitle{Aussagen: Verknüpfung}


\begin{KITexampleblock}{ }
\begin{itemize}
\item Die Verknüpfung $A \rightarrow B$ entspricht der Verknüpfung $\neg A \vee B$
\end{itemize}
\end{KITexampleblock}\\
\ \\
\begin{KITexampleblock}{ }
\begin{itemize}
\item Die Verknüpfung $A \leftrightarrow B$ entspricht der Verknüpfung $(A \rightarrow B) \wedge (B \rightarrow A)$
\end{itemize}
\end{KITexampleblock}\\
\ \\
\begin{KITexampleblock}{ }
\begin{itemize}
\item Die Reihenfolge der Verknüpfungen ist: (, ) vor $\neg$ vor $\vee$ vor $\wedge$ vor $\rightarrow$ vor $\leftrightarrow$
\end{itemize}
\end{KITexampleblock}\\


\end{frame}


%Lösung
\begin{frame}
\frametitle{Aufgabe}
Wertet die komplexe Aussage mittels Tabelle aus:\\
\ \\
$\neg A \wedge (B \rightarrow A) \leftrightarrow \neg (A \vee \neg B)$\\
\ \\

\begin{tabular}{cc|ccccc|c|cccc}
&&1 & 9 & 2 & 6 & 3 & 10 & 8 & 4 & 7 & 5\\
A & B & $\neg A$ & $\wedge$ & $(B$ & $\rightarrow$ & $A)$ & $\leftrightarrow$ & $\neg$ & $(A$ & $\vee$ & $\neg B)$\\
\hline
f & f & w & w & f & w & f & f & f & f & w & w\\
f & w & w & f & w & f & f & f & w & f & f & f\\
w & f & f & f & f & w & w & w & f & w & w & w\\
w & w & f & f & w & w & w & w & f & w & w & f\\
\end{tabular}\\

\end{frame}


%Quantoren
\begin{frame}
\frametitle{Quantoren}
es gibt 2 Quantoren:
\ \\
\begin{itemize}
\item $\forall$: Für alle/jedes ... gilt, dass...
\item $\exists$: Es gibt (min.) ein..., sodass...
\end{itemize}
\end{frame}


%Lösung
\begin{frame}
\frametitle{Aufgabe}
Formulert die Aussagen mittels Prädikatenlogik (d.h. mit Quantoren und Konnektiven):\\
\ \\
Sei V die Menge aller Vögel, Eltern(v) die Menge der Eltern des Vogels v, Farbe(v) die Farbe von v\\
\ \\
\begin{itemize}
\item Jeder Vogel kann fliegen, wenn alle Eltern fliegen können
\begin{itemize}
\item $\forall v \in V:$ Eltern(v) können fliegen $\rightarrow$ v kann fliegen.
\end{itemize}
\item Alle schwarzen Vögel können fliegen
\begin{itemize}
\item $\forall v \in V:$ Farbe(v)$ = $schwarz $\rightarrow$ v kann fliegen.
\end{itemize}
\end{itemize}
\end{frame}


%Interpretation
\begin{frame}
\frametitle{Interpretationen}
\begin{KITexampleblock}{Definition}
\begin{itemize}
\item eine Interpretation I ist eine Belegung der Variablen in der Formel\\
z.B. I(A)=w und I(B)=f
\end{itemize}
\end{KITexampleblock}\\
\ \\
$val_I(F)$ gibt den Wahrheitswert der Formel F zur Belegung I an.\\
z.B. $val_I(A \rightarrow B)=f$ mit obigem I\\
\ \\
eine aussagenlogische Formel ist entweder

\begin{itemize}
\item \textbf{Nicht erfüllbar}, wenn es keine Interpretation gibt für die sie wahr ist.
\item \textbf{Erfüllbar}, wenn es eine Interpretation gibt für die sie erfüllbar ist.
\item \textbf{Nicht Allgemeingültig}, wenn sie für nicht jede Interpration wahr ist (also für min. 1 Interpretation falsch)
\item \textbf{Allgemeingültig}, wenn sie für jede Interpretation wahr ist.
\end{itemize}
\end{frame}

\end{document}
