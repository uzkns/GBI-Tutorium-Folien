\documentclass[ngerman,hyperref={pdfpagelabels=false}]{beamer}

% -----------------------------------------------------------------------------

\graphicspath{{images/}}

% -----------------------------------------------------------------------------

\usetheme{KIT}

\setbeamercovered{transparent}
%\setbeamertemplate{enumerate items}[ball]

\newenvironment<>{KITtestblock}[2][]
{\begin{KITcolblock}<#1>{#2}{KITblack15}{KITblack50}}
{\end{KITcolblock}}

\usepackage[ngerman,english]{babel}
\usepackage[utf8]{inputenc}
\usepackage[TS1,T1]{fontenc}
\usepackage{array}
\usepackage{multicol}
\usepackage[absolute,overlay]{textpos}
\usepackage{beamerKITdefs}
\usepackage{amsfonts}

\newcommand{\code}[1]{\texttt{#1}}


\pdfpageattr {/Group << /S /Transparency /I true /CS /DeviceRGB>>}	%required to prevent color shifting withd transparent images


\title{Tutorium 17, \#12}
\subtitle{Max Göckel-- \textit{uzkns@student.kit.edu}}

\author[Max Göckel]{Max Göckel}
\institute{Institut für Theoretische Informatik - Grundbegriffe der Informatik}

\TitleImage[width=\titleimagewd,height=\titleimageht]{titel}

\KITinstitute{Institut f\"ur Theoretische Informatik}
\KITfaculty{Fakult\"at f\"ur Informatik}

% -----------------------------------------------------------------------------

\begin{document}
\setlength\textheight{7cm} %required for correct vertical alignment, if [t] is not used as documentclass parameter


% title frame
\begin{frame}
  \maketitle
\end{frame}


%Zählautomat
\begin{frame}
  \frametitle{Automaten}
(Endliche) Automaten sind ähnlich zu kleinen Maschinen die mit einer Eingabe häufig ihren Zustand ändern und dann damit eine Ausgabe formen.\\
\ \\
Als Beispiel haben wir einen Zählautomaten, der die Tasten "$+$", "$-$" sowie "$R$" und "$OK$" hat. Der Automat zählt maximal von 0 bis 5.\\
\begin{itemize}
	\item Mit "$+$" wird der Wert um 1 erhöht (max. 5)
	\item Mit "$-$" wird der Wert um 1 reduziert (min. 0)
	\item Mit "$R$" wird der Zähler wieder auf 0 gesetzt (egal von wo aus)
	\item Mit "$OK$" wird der Wert ausgegeben und der Zähler zurückgesetzt
\end{itemize}
\end{frame}


%Grafik
\begin{frame}
  \frametitle{Automaten}
An der Tafel.
\end{frame}


%Zustandsübergänge
\begin{frame}
  \frametitle{Automaten: Zustandsübergang}
\begin{itemize}
	\item $R$, $OK$: Eingaben
	\item $\epsilon$, $4$: Ausgaben, dabei ist $\epsilon =$ Keine Ausgabe
	\item Kreis mit der 4: Zustand, in unserem Fall der Zähler
	\begin{itemize}
		\item 0 ist bei unserem Zählautomat der Anfangszustand, markiert mit einem Pfeil aus dem nichts
	\end{itemize}
	\item Pfeil ($\rightarrow$): Zustandsübergang von-nach, ähnlich einer gerichteten Kante
\end{itemize}
\end{frame}


%Mealy Automat
\begin{frame}
  \frametitle{Automaten: Mealy-Automat}
Ein Mealy-Automat ist ein 6-Tupel aus Mengen, Funktionen, einem Startzustand und zwei Alphabeten.\\
$A=( Z, z_0, X, f, Y, g )$ \\
\ \\
\begin{enumerate}
	\item $Z$ ist die (endliche) Zustandsmenge
	\item $z_0$ ist der Startzustand
	\item $X$ ist das Eingabealphabet
	\item $f: Z \times X \rightarrow Z$ ist die Zustandsübergangsfunktion
	\begin{itemize}
		\item Aus einem Tupel $(z_1, x_1)$ bestehend aus Zustand und Eingabe wird der nächste Zustand ($\rightarrow z_2$) abgebildet
	\end{itemize}
	\item $Y$ ist das Ausgabealphabet
	\item $g: Z \times X \rightarrow Y^*$ ist die Ausgabefunktion
	\begin{itemize}
		\item Aus einem Tupel $(z_1, x_1)$ bestehend aus Zustand und Eingabe wird das Ausgabewort ($\rightarrow y_1$) abgebildet
	\end{itemize}
\end{enumerate}
\end{frame}


%Funktionen
\begin{frame}
  \frametitle{Mealy-Automat: Funktionen}
Es gibt 4 Funktionen, die man auf einen Mealy-Automaten anwenden kann: \\
\begin{enumerate}
	\item Mit $f^*$: kann man ein Wort $w \in X^*$ (und ggf. einen Zustand) eingeben und erhält den Endzustand
	\item Mit $f^{**}$ kann man ein Wort eingeben und erhält alle durchlaufenen Zustände
	\item Mit $g^*$ kann für ein Wort die letzte Ausgabe angegeben werden
	\item Mit $g^{**}$ kann für ein Wort alle Ausgaben angegeben werden
\end{enumerate}
\ \\
(Die 4 Funktionen sind Homomorphismen)
\end{frame}


%Moore Automat
\begin{frame}
  \frametitle{Automaten: Moore-Automat}
Ein Moore-Automat ist auch ein 6-Tupel aus Mengen, Funktionen, einem Startzustand und zwei Alphabeten.\\
$B=( Z, z_0, X, f, Y, g )$ \\
\ \\
Im Gegensatz zum Mealy-Automat hat der Moore-Automat seine Ausgabe nicht im Übergang, sondern direkt im Zustand, also $h: Z \rightarrow Y^*$.\\
\ \\
Auch beim Moore-Automaten gelten die 4 Funktionen, $f^*$, $f^{**}$ sind identisch, aber $h^*$, $h^{**}$ ersetzten $g^*$, $g^{**}$.\\
Effektiv ändert sich aber nichts an der Funktionsweise der Funktionen.
\end{frame}


%Aufgaben
\begin{frame}
  \frametitle{Automaten: Aufgaben}
Gib an:\\
\ \\
\begin{itemize}
	\item $f^*(z_1, STSTST)$
	\item $f^*(z_2, TTTSSS)$
	\item $h^{**}(SSTSTT)$
	\item $h^{**}(z_1, \epsilon)$
	\item $h^*(z_3, TSSSSS)$
	\item $h^*(z_1, STST)$
\end{itemize}
\end{frame}


%LÖSUNG
\begin{frame}
  \frametitle{Automaten: Lösungen}
Gib an:\\
\ \\
\begin{itemize}
	\item $f^*(z_1, STSTST) = z_3$
	\item $f^*(z_2, TTTSSS) = z_2$
	\item $h^{**}(SSTSTT) = abacabb$
	\item $h^{**}(z_1, \epsilon) = b$
	\item $h^*(z_3, TSSSSS) = a$
	\item $h^*(z_1, STST) = b$
\end{itemize}
\end{frame}


%Akzeptoren
\begin{frame}
  \frametitle{Endliche Akzeptoren}
Endliche Akzeptoren sind Sonderfälle von Moore-Automaten die nur das Ausgabealphabet $Y = \{0, 1\}$ haben. \\
Zustände mit Ausgabe 1 umkreist man doppelt, Zustände mit Ausgabe 0 umkreist man einfach, um sich die Ausgabe zu sparen.\\
\ \\
Als Tupel ist ${EA} = (Z, z_0, X, f, F)$ mit $F =$ "Menge der Zustände mit Ausgabe 1" ein endlicher Akzeptor.
\end{frame}


%Akzeptoren und Sprachen
\begin{frame}
  \frametitle{Endliche Akzeptoren und Sprachen}
$q_4$ ist ein akzeptierender Zustand, d.h. der Akzeptor akzeptiert genau alle $w \in X^*$ die am Ende in $q_4$ landen.\\
Mathematisch: $f^*(w) \in F \Leftrightarrow$ \emph{"EA akzeptiert w"} \\
\ \\
Die Menge aller Wörter $w \in X^*$ die vom Akzeptor $A$ akzeptiert werden ergibt dabei die Sprache $L(A)$.
\end{frame}


%Aufgabe
\begin{frame}
  \frametitle{Endliche Akzeptoren und Sprachen}
\begin{enumerate}
	\item Welche Sprache akzeptiert der Akzeptor $A$ an der Tafel?
	\item Zeichne den Akzeptor $B$ mit $L(B) = \{ a \} { \{ ab \} }^* \{ b \}$
\end{enumerate}
\end{frame}
























\end{document}
