\documentclass[ngerman,hyperref={pdfpagelabels=false}]{beamer}

% -----------------------------------------------------------------------------

\graphicspath{{images/}}

% -----------------------------------------------------------------------------

\usetheme{KIT}

\setbeamercovered{transparent}
%\setbeamertemplate{enumerate items}[ball]

\newenvironment<>{KITtestblock}[2][]
{\begin{KITcolblock}<#1>{#2}{KITblack15}{KITblack50}}
{\end{KITcolblock}}

\usepackage[ngerman,english]{babel}
\usepackage[utf8]{inputenc}
\usepackage[TS1,T1]{fontenc}
\usepackage{array}
\usepackage{multicol}
\usepackage[absolute,overlay]{textpos}
\usepackage{beamerKITdefs}
\usepackage{amsfonts}

\newcommand{\code}[1]{\texttt{#1}}


\pdfpageattr {/Group << /S /Transparency /I true /CS /DeviceRGB>>}	%required to prevent color shifting withd transparent images


\title{Tutorium 42, \#8}
\subtitle{Max Göckel-- \textit{uzkns@student.kit.edu}}

\author[Max Göckel]{Max Göckel}
\institute{Institut für Theoretische Informatik - Grundbegriffe der Informatik}

\TitleImage[width=\titleimagewd,height=\titleimageht]{titel}

\KITinstitute{Institut f\"ur Theoretische Informatik}
\KITfaculty{Fakult\"at f\"ur Informatik}

% -----------------------------------------------------------------------------

\begin{document}
\setlength\textheight{7cm} %required for correct vertical alignment, if [t] is not used as documentclass parameter


% title frame
\begin{frame}
  \maketitle
\end{frame}


%Prädikatenlogik
\begin{frame}
  \frametitle{Prädikatenlogik: Grundlagen}

Mit der Prädikatenlogik kann man mehr Aussagen formulieren als mit der Aussagenlogik. Wir können damit auch Aussagen in prädikatenlogische Formeln \emph{("Mathesprache")} umformlieren.\\
\ \\
Die Prädikatenlogik (PL) 1. Stufe benutzt die folgenden Zeichen:\\
  \begin{itemize}
	\item $\neg, \vee, \wedge, \rightarrow, \leftrightarrow, (, )$ aus der Aussagenlogik
	\item $\forall$ (Allquantor für Variablen) und $\exists$ (Existenzquantor für Variablen)
	\item $\dot = $ (Objektgleichheit/ Gleichheit 1. Stufe)
	\item Variablen (zum Beispiel x, y, z)
	\item Funktionen (zum Beispiel f, g, h)
	\item Prädikate (zum Besipiel p, q, r)
  \end{itemize}
\end{frame}


%Prädikate
\begin{frame}
  \frametitle{Prädikatenlogik: Prädikate}
Prädikate sind Funktionen, die für eine Eingabe (eine oder mehr Variablen) einen Wahrheitswert zurückgeben.\\
Ähnlich wie zum Beispiel \code{public boolean p (Var x, Var y)} in Java, die für Var x und y einen Boolean-Wahrheitswert zurückgibt.\\
\ \\
Prädikate in PL sehen zum Besipiel so aus:\\
\begin{itemize}
	\item Vater(x,y): Wahr, wenn x Vater von y ist
	\item Mann(x): Wahr wenn x Männlich ist, sonst falsch
	\item Verheiratet(x,y): wahr wenn x und y verheiratet sind
\end{itemize}
\end{frame}


%Prädikatenlogische Formeln Aufgaben
\begin{frame}
  \frametitle{Prädikatenlogik: Aufgaben}
Die folgenden Prädikate sind gegeben\\
\begin{itemize}
	\item Vater(x,y): Wahr, wenn x Vater von y ist
	\item Mann(x): Wahr wenn x Männlich ist, sonst falsch
	\item Verheiratet(x,y): wahr wenn x und y verheiratet sind
\end{itemize}
\ \\
Drücke die Aussagen in PL aus\\
\begin{itemize}
	\item Jede männliche Person hat mindestens einen Vater
	\item Jeder Mann hat mehrere Kinder
	\item Jede Frau ist mit \emph{höchstens} einem Mann verheiratet
\end{itemize}
\end{frame}


%LÖSUNG
%Prädikatenlogische Formeln Aufgaben
\begin{frame}
  \frametitle{Prädikatenlogik: Lösung}
Drücke die Aussagen in PL aus\\
\begin{itemize}
	\item Jede männliche Person hat mindestens einen Vater
	\begin{itemize}
		\item $\forall x \exists y (Mann(x) \vee Vater(y,x))$
		\item "Für jede Person x exist. min. 1 Person y wo gilt: x ist ein Mann und y ist Vater von x" 
	\end{itemize}
	\item Jeder Mann hat mehrere Kinder
	\begin{itemize}
		\item $\forall x \exists y \exists z (Mann(x) \wedge Vater(x,y) \wedge Vater(x,z) \wedge \neg (y \dot = z))$
		\item "Für jede Person x existiert min. 1 Person y und 1 Person z wo gilt: x ist männlich und Vater von y und z wobei y $\neq$ z ist"
	\end{itemize}
	\item Jede Frau ist mit \emph{höchstens} einer Person verheiratet
	\begin{itemize}
		\item $\forall x \forall y \forall z (\neg Mann(x) \wedge \neg (y \dot = z) \wedge Verheiratet(x,y)) \rightarrow \neg Verheiratet(x,z))$
		\item Für jede Person x,y, z gilt: Ist x nicht männlich und y,z männlich und y $\neq$ z und x mit y verheiratet so folgt: x ist nicht mit z verheiratet
	\end{itemize}
\end{itemize}
\end{frame}


%PL Aufbau
\begin{frame}
  \frametitle{Prädikatenlogik: Aufbau}
\begin{itemize}
	\item Jede Variable ist ein Term (Bsp.: x)
	\item Eine Funktion f mit Termen als Eingabe ist wieder ein Term (Bsp: f(x,y, f(z)))
\end{itemize}
\ \\
\begin{itemize}
	\item Atomare Formeln sind die "kleinsten" PL-Formeln die wir bilden können
	\item Objektgleichheit $(x \dot = y)$ ist eine atomare Formel
	\item Ein Prädikat mit Termen als Eingabe ist eine atomare Formel
	\item Aus atomaren Formeln bilden wir mittels Konnektiven größere PL-Formeln
\end{itemize}
\ \\
Natürlich gibt es auch bei PL Regeln für (syntaktisch) korrekte Formeln.
\end{frame}


%Aufgabe Bildung von PL-Formeln
\begin{frame}
  \frametitle{Prädikatenlogik: Aufgabe}
\emph{p,q} einstellige Prädikate, \emph{r} zweistelliges Prädikat, \emph{c,d} Konstanten (nullstellige funktionen), \emph{x,y,z} Variablen\\
\ \\
Sind die folgenden PL-Formeln syntaktisch korrekt gebildet?\\
\begin{itemize}
	\item $\forall x ( \forall y ( \forall z (r(x,y) \wedge r(y,z) \rightarrow r(x,z) ) ) )$
	\item $\forall x \forall y (p(x) \rightarrow r(y))$
	\item $\forall x \exists p: p(x)$
	\item $\forall x (q(x)) \wedge \exists x (\neg q(x))$
\end{itemize}
\end{frame}


%LÖSUNG
\begin{frame}
  \frametitle{Prädikatenlogik: Aufgabe}
Sind die folgenden PL-Formeln syntaktisch korrekt gebildet?\\
\begin{itemize}
	\item $\forall x ( \forall y ( \forall z (r(x,y) \wedge r(y,z) \rightarrow r(x,z) ) ) )$ - Ja
	\item $\forall x \forall y (p(x) \rightarrow r(y))$ - Nein, r ist zweistellig
	\item $\forall x \exists p: p(x)$ - Nein, $\exists$ gilt nur für Variablen, nicht für Prädikate
	\item $\forall x (q(x)) \wedge \exists x (\neg q(x))$ - Ja
\end{itemize}
\end{frame}


%PL Wirkungsbereich
\begin{frame}
  \frametitle{Prädikatenlogik: Quantoren-Wirkungsbereich}
$\forall$ und $\exists$ haben einen begrenzten "Wirkungsbereich" auf die Variablen hinter ihnen.\\
Der Wirkungsbereich eines Quantores ist eine Teilformel und die Var. muss die Bindung in der Teilformel befolgen.\\
\ \\
Bspe.:\\
\begin{itemize}
	\item $\forall x (p(x) \wedge q(x,y) ) \rightarrow r(x)$ - beide Prädikate sind gebunden
	\item $\forall x (p(x) \wedge \forall x (\neg p(x) ) )$ - Überschattung, also ein x umbenennen
\end{itemize}
\ \\
Die Bindung von Variablen wird später noch sehr wichtig.
\end{frame}


%PL Bindung Aufgabe
\begin{frame}
  \frametitle{Prädikatenlogik: Aufgabe}
Kennzeichne die Bindung von Variablen in der Formel mit Pfeilen zu ihren Quantoren\\
\begin{itemize}
	\item $p(x) \rightarrow \forall x \exists y (p(x) \wedge q(y,z) \leftrightarrow \forall z ( q(x,z) ) )$
\end{itemize}
\end{frame}


%PL Substitution
\begin{frame}
  \frametitle{Prädikatenlogik: Substitution}
In PL-Formeln kann man Variablen gegen andere Variablen oder Terme ersetzen.\\
Wichtig dabei: Es werden nur freie Variablen ersetzt (also solche die nicht an einen Quantor gebunden sind)\\
\ \\
Eine Substitution schreibt man so: $\epsilon [x / y](A)$, sie ersetzt alle x durch y in Formel A.\\
\ \\
Beispielsubstitutionen:
\begin{itemize}
	\item $\beta_1 [x / 5] ( p(x) \vee q(x,y) ) = ( p(5) \vee q(5,y) )$
	\item $\beta_2 [x / f(x)](p(x) \vee \forall x (q(x, y))) = p(f(x)) \vee \forall x (q(x, y)))$
\end{itemize}
\end{frame}


%Substitution Kollision
\begin{frame}
  \frametitle{Prädikatenlogik: Substitutionskollisionen}
Eine Substitution enthält eine Kollision falls in der Substitution eine freie Variable gegen eine gebundene Variable (also eine Variable im Wirkungsbereich eine Quantors) getauscht wird.\\
Eine Substitution, die keine Kollision enthält heißt "Kollisionsfrei" und ist zu bevorzugen.\\
\ \\
Bsp.:\\
\begin{itemize}
	\item $\beta_1 [y / x] ( \forall x : x \leftrightarrow y ) = ( \forall x : x \leftrightarrow x )$
\end{itemize}
\end{frame}

%Aufgabe Substiution
\begin{frame}
  \frametitle{Prädikatenlogik: Aufgabe}
Führe die Substitutionen aus:\\
\begin{itemize}
	\item $\sigma_1 [ x / f(y,z), y / 9 ](q(x, f(y,z)) \wedge \forall x (p(x)) )$
	\item $\sigma_2 [ x / g(y), y / x ]( \beta_1 [ x/y, y/f(x,y) ] (\exists z ( r(x,y,z) ) ) )$
\end{itemize}
\end{frame}


%LÖSUNG
\begin{frame}
  \frametitle{Prädikatenlogik: Lösung}
Führe die Substitutionen aus:\\
\begin{itemize}
	\item $\sigma_1 [ x / f(y,z), y / 9 ](q(x, f(y,z)) \wedge \forall x (p(x)) )$
	\begin{itemize}
		\item $q(f(y,z), f(9, z) \wedge \forall x ( p(x) )$
	\end{itemize}
	\item $\sigma_2 [ x / g(y), y / x ]( \beta_1 [ x/y, y/f(x,y) ] (\exists z ( r(x,y,z) ) ) )$
	\begin{itemize}
		\item $\sigma_2 [ x / g(y), y / x ]( \exists z ( r(y, f(x,y), z) ) )$
		\item $\exists z ( r(x,f( g(y), x), z) )$
	\end{itemize}
\end{itemize}
\end{frame}


%PL Semantik
\begin{frame}
  \frametitle{Prädikatenlogik: Semantik}
Neben dem syntaktisch korrekten Aufbau von Prädikatenlogik-Formeln kann man diesen auch eine Bedutung (Semantik) zuordnen und sie auswerten.\\
\ \\
Ähnlich in der Aussagnelogik gibt es \emph{Val} und \emph{I}. I weist einer Konstanten, Funktion, Abbildung oder Relation eine Bedeutung zu, wobei Val einen ganzen Term auswertet.\\
\ \\
Zusätzlich gibt es in der PL noch 2 weitere Dinge, die sich von der Aussagenlogik unterscheiden:\\
\begin{itemize}
	\item Die Zielmenge, auf die I abbildet (Ein Universum \emph{D} und $\{w,f\}$)
	\item Belegung ungebundener bzw. freier Variablen (idR $\beta$)
\end{itemize}
\ \\
Das Tupel (D,I) ist damit eine Interpretation und mit $val_{D,I, \beta}$ kann betimmt werden ob eine Formel wahr ist.
\end{frame}


%PL Aufgabe
\begin{frame}
  \frametitle{Prädikatenlogik: Aufgabe}
p, einstelliges Prädikat, q zweistelliges Prädikat, f zweistellige Funktion, c, Konstante, x,y Variablen.\\
\ \\
D = $\mathbb{N}_0$\\
$\beta(x) = 7$\\
I(p(x)) = w $\leftrightarrow x \geq 5$\\
I(q(x,y)) = w $\leftrightarrow x > y$\\
I(c) = 10\\
I(f(x,y)) = (x-y)\\
\ \\
\begin{itemize}
	\item $p(x) \rightarrow \exists y ( q(y,x) \wedge p(y) )$
	\item $p(x) \rightarrow \exists y ( q( f(c,y), x ) \wedge p(y) )$
\end{itemize}
\end{frame}



%LÖSUNG
\begin{frame}
  \frametitle{Prädikatenlogik: Lösung}
\begin{itemize}
	\item $p(x) \rightarrow \exists y ( q(y,x) \wedge p(y) )$
	\begin{itemize}
		\item $Val_{D,I,\beta}(p(x)) = I(p(7)) = w$
		\item $\exists y$ bel. wählbar also $y = 8 \Rightarrow I(q(8,7)) = w$ und $I(p(8)) = w \Rightarrow Val_{D,I,\beta}(\exists y ...) = w \Rightarrow$
$Val_{D,I,\beta}(p(x) \rightarrow ...) = w$
	\end{itemize}
	\item $p(x) \rightarrow \exists y ( q( f(c,y), x ) \wedge p(y) )$
	\begin{itemize}
		\item $Val_{D,I,\beta}(p(x)) = w$ (s.o.)
		\item $Val_{D,I, \beta}(q( f(c,y), x ) ) = Val_{D,I, \beta}(q( f(c,y), x ) ) = Val_{D,I, \beta}(q( 10-y, 7 ) ) \rightarrow w$ für $y \in \{ 0,1,2 \}$
		\item $Val_{D,I, \beta}( p(y) ) = w$ für $y \geq 5$
		\item Keine Lösung da $y \in \{ 0,1,2 \}$ und $y \geq 5$ nicht beide gelten können
	\end{itemize}
\end{itemize}
\end{frame}














































\end{document}
