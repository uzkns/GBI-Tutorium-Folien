\documentclass[ngerman,hyperref={pdfpagelabels=false}]{beamer}

% -----------------------------------------------------------------------------

\graphicspath{{images/}}

% -----------------------------------------------------------------------------

\usetheme{KIT}

\setbeamercovered{transparent}
%\setbeamertemplate{enumerate items}[ball]

\newenvironment<>{KITtestblock}[2][]
{\begin{KITcolblock}<#1>{#2}{KITblack15}{KITblack50}}
{\end{KITcolblock}}

\usepackage[ngerman,english]{babel}
\usepackage[utf8]{inputenc}
\usepackage[TS1,T1]{fontenc}
\usepackage{array}
\usepackage{multicol}
\usepackage[absolute,overlay]{textpos}
\usepackage{beamerKITdefs}
\usepackage{amsfonts}


\pdfpageattr {/Group << /S /Transparency /I true /CS /DeviceRGB>>}	%required to prevent color shifting withd transparent images


\title{Tutorium 17, \#3}
\subtitle{Max Göckel-- \textit{uzkns@student.kit.edu}}

\author[Max Göckel]{Max Göckel}
\institute{Institut für Theoretische Informatik - Grundbegriffe der Informatik}

\TitleImage[width=\titleimagewd,height=\titleimageht]{titel}

\KITinstitute{Institut f\"ur Theoretische Informatik}
\KITfaculty{Fakult\"at f\"ur Informatik}

\newcommand{\zz}{\mathrm{Z\kern-.3em\raise-0.5ex\hbox{Z}}}

% -----------------------------------------------------------------------------

\begin{document}
\setlength\textheight{7cm} %required for correct vertical alignment, if [t] is not used as documentclass parameter


% title frame
\begin{frame}
  \maketitle
\end{frame}


%%%

%Beweis von Aussagen
\begin{frame}
\frametitle{Aussagen beweisen}
In sonstigen Naturwissenschaften (Biologie, Chemie, ...): Versuch oft genug durchführen, wenn sich das Ergebnis sich während des Versuches nicht ändert ist es wohl richtig.\\
\textbf{Lösung: Ein Experiment mehrfach und in verschiedenen Zuständen durchführen}\\
\ \\
In der Mathematik und Informatik: Beweis von Aussagen für unendlich viele Zustände (am besten: \emph{alle.})\\
\textbf{Aber wie?}
\end{frame}


%Induktion
\begin{frame}
\frametitle{Vollständige Induktion}
Lösung für das Problem ist die \emph{vollständige Induktion.}\\
\ \\
z.B.: Zeige dass $\forall n \in \mathbb{N}_+: \exists m \in \mathbb{N}_0: m < n$.\\
Möglich, da $n$ durchzählbar sind (1, 2, 3,..)\\
\ \\
\begin{KITexampleblock}{Überlegung}
	\begin{itemize}
		\item An n=0 n anfangen, Behauptung zeigen, weiterzählen, somit Behauptung für alle n zeigen.
	\end{itemize}
	\end{KITexampleblock}
\end{frame}


%Vorgehen
\begin{frame}
\frametitle{Induktion: Vorgehen}
Drei Schritte:\\
\ \\
\begin{enumerate}
\item Induktionsanfang (IA): Den kleinsten Wert nehmen und die Behauptung für diesen zeigen. Manchmal noch die Behauptung für den ersten Schritt zeigen.
\item Induktionsvoraussetzung: Die Behauptung <Behauptung> gilt für ein beliebiges aber festes $n \in \mathbb{N}_+$ (oder worüber man die Induktion anwendet).
\item Induktionsschritt (IS): wenn die Behauptung für $n$ gilt, soll sie auch für $n+1$ gelten. Das zeigen wir jetzt.
\end{enumerate}
\end{frame}


%Induktionsschritt im Detail
\begin{frame}
\frametitle{Induktionsschritt: Vorgehen}
Der Induktionsschritt soll zeigen, dass unsere Behauptung für $n+1$ gilt, wenn sie für $n$ gilt.\\
\ \\
\begin{enumerate}
\item $n+1$ in die Behauptung einsetzen.
\item Neue Behauptung so umformen, dass Behauptung mit $n$ wieder "auftaucht"...
\item Nach der IV gilt die Aussage für unser $n$ welches gerade "aufgetaucht" ist...
\item Die aufgelöste Aussage in den Induktionsschritt einsetzen...
\item noch etwas umformen und den "$n+1$"-Fall zeigen.
\item Freuen :)
\end{enumerate}
\end{frame}


%AUFGABEN
\begin{frame}
\frametitle{Induktion: Aufgaben}

\begin{itemize}
\item $\zz$ dass $n^2 + n \forall n \geq 0$ gerade ist 
\item $\zz$ (1+2+3+...+n)= $ \sum \limits_{i=1}^n i = \frac{n(n+1)}{2}, \forall n \geq 1$
\end{itemize}
\ \\
Die Lösungen und weitere Aufgaben sind im ILIAS.
\end{frame}


%Formale Sprachen
\begin{frame}
\frametitle{Formale Sprachen}

Sprache: Aussprache, Stil, Satzbau, Wortwahl\\
\ \\
In der Informatik: Aufbau vom Befehlen, Compiler, WWW-Seiten\\
\ \\
\begin{KITalertblock}{Problem}
	\begin{itemize}
		\item Woher weiß der Computer ob das (Sprach-)Gebilde korrekt ist?
	\end{itemize}
\end{KITalertblock}
\end{frame}


%Formale Sprachen
\begin{frame}
\frametitle{Formale Sprachen}

Sprache: Aussprache, Stil, Satzbau, Wortwahl\\
\ \\
In der Informatik: Aufbau vom Befehlen, Compiler, WWW-Seiten\\
\ \\
\begin{KITinfoblock}{Lösung}
\begin{itemize}
	\item Eine formale Sprache als Teilmenge von $A^*$ definiert was richtig ist und was nicht
\end{itemize}
\end{KITinfoblock}
\end{frame}


%Beispiel
\begin{frame}
\frametitle{Formale Sprachen: Beispiel}

$A = \{0, 1, 2, 3, 4, 5, 6, 7, 8, 9\} \cup \{., -, +\}$, F $\subseteq A^*$ Formalsprache der Dezimaldarstellung aller Zahlen $\in \mathbb{Q}$
\end{frame}

%Beispiel
\begin{frame}
\frametitle{Formale Sprachen: Beispiel}

$A = \{0, 1, 2, 3, 4, 5, 6, 7, 8, 9\} \cup \{., -, +\}$, F $\subseteq A^*$ Formalsprache der Dezimaldarstellung aller Zahlen $\in \mathbb{Q}$

\begin{itemize}
\item +1234567890.0
\item +236.1
\item -310.25
\end{itemize}
\ \\
\begin{itemize}
\item +-5
\item 3+
\item 31..
\item -.+.-.+.-.+.-.+.-
\end{itemize}
\end{frame}


%Beispiel
\begin{frame}
\frametitle{Formale Sprachen: Beispiel}

$A = \{0, 1, 2, 3, 4, 5, 6, 7, 8, 9\} \cup \{., -, +\}$, F $\subseteq A^*$ Formalsprache der Dezimaldarstellung aller Zahlen $\in \mathbb{Q}$

\begin{enumerate}
	\item Plus oder Minus (+/-)
	\item Mindestens eine Ziffer (0..9)
	\item Dezimalpunkt (.)
	\item Mindestens eine Ziffer (0..9)
\end{enumerate}
\ \\
\begin{itemize}
	\item +1234567890.0
	\item +236.1
	\item -310.25
\end{itemize}

\end{frame}


%Beispiel
\begin{frame}
\frametitle{Formale Sprachen: Beispiel}

$A = \{0, 1, 2, 3, 4, 5, 6, 7, 8, 9\} \cup \{., -, +\}$, $F \subseteq A^*$ Formalsprache der Dezimaldarstellung aller Zahlen $\in \mathbb{Q}$

\begin{enumerate}
	\item Plus oder Minus (+/-)
	\item Mindestens eine Ziffer (0..9)
	\item Dezimalpunkt (.)
	\item Mindestens eine Ziffer (0..9)
\end{enumerate}
\ \\

$A_{num} = \{0, 1, 2, 3, 4, 5, 6, 7, 8, 9 \} \subset A$ \\
\ \\
$F = \{ + \cdot w_1 \cdot . \cdot w_2 | w_1, w_2 \in A_{num}^* \wedge |w_1|, |w_2| \geq 1 \} \cup$ \\
$\{ - \cdot w_1 \cdot . \cdot w_2 | w_1, w_2 \in A_{num}^* \wedge |w_1|, |w_2| \geq 1 \}$\\
\end{frame}


%Lösung
\begin{frame}
\frametitle{Formale Sprachen: Aufgaben}
Wie sehen Wörter aus den Sprachen aus?\\
\ \\
\begin{itemize}
\item $L_1$ = \{$w_1 ab w_2 | w_1, w_2 \in \{\ a,b\}^*$\}
	\begin{itemize}
	\item Alle Wörter aus $A$ die Das Teilwort ab enthalten (z.B. ab, aaaab, bababab, aaaaabbb)
	\end{itemize}
\item $L_2$ = \{ $w_1 w_2 | w_1 \in \{a\}^* \wedge w_2 \in \{b\}^*$ \}
	\begin{itemize}
	\item Beliebige Anzahl an a's gefolgt von einer beliebigen Zahl an b's (z.B. aaaab, abb, aaaaaabbbbb)
	\end{itemize}
\item $L_3$ = \{$ w_1 w_2 | w_1, w_2 \in \{a,b\}^* |w_1| = 0 \wedge |w_2|<|w_1|$\}
	\begin{itemize}
	\item Nichts, da $|w_2| < 0$ nicht möglich ist
	\end{itemize}
\end{itemize}
\end{frame}


%Produkt
\begin{frame}
\frametitle{Formale Sprachen: Produkt}

\begin{KITexampleblock}{Definition}
	\begin{itemize}
		\item Seien $F_1, F_2$ formale Sprachen über $A$, so ist das Produkt $L_1 * L_2$ = \{$w_1 w_2 | w_1 \in L_1 \wedge w_2 \in L_2 $\}
	\end{itemize}
	\end{KITexampleblock}\\
\ \\
Zum Beispiel mit $L_1$ = \{a, aa, ab\}, $L_2$ = \{b, ba, bb\}:\\
\ \\
\begin{itemize}
\item $L_1 * L_2$ = \{ab, aab, ab, aaba, aabb, abb, abba, abbb\}
\item $L_2 * L_1$ = \{ba, baa, bab, baa, baaa, baab, bba, bbab\}
\end{itemize}

\end{frame}


%Potenz von Sprachen
\begin{frame}
\frametitle{Formale Sprachen: Potenz}

\begin{KITexampleblock}{Definition}
	\begin{itemize}
		\item Sei $F_1$ formale Sprache über $A$, so ist die Potenz $L_1^n$ die $n$-fache Verkettung von $L_1$ mit sich selbst
	\end{itemize}
	\end{KITexampleblock}\\
\ \\
Zum Beispiel mit $L_1$ = \{a, b\}
\ \\
\begin{itemize}
\item $L_1^0$ = \{$ \epsilon $\}
\item $ L_1^2$ = \{aa, ab, ba, bb\}
\end{itemize}
\end{frame}


%Konkatenationsabschluss
\begin{frame}
\frametitle{Formale Sprachen: Konkatenationsabschluss}

\begin{KITexampleblock}{Definition}
	\begin{itemize}
		\item Sei $F$ formale Sprache über $A$, so ist $F^* = \bigcup\limits_{i \in \mathbb{N_0}} L^i$ der Konkatenationsabschluss von $F$
	\end{itemize}
	\end{KITexampleblock}\\
\ \\
Jede unendlich häufige Konkatenation von Wörtern aus $F$ liegt in $F^*$.
\end{frame}


%Epsilonfreier Konkatenationsabschluss
\begin{frame}
\frametitle{Formale Sprachen: $\epsilon$-freier Konkatenationsabschluss}

\begin{KITexampleblock}{Definition}
	\begin{itemize}
		\item Sei $F$ formale Sprache über $A$, so ist $F^+ = \bigcup\limits_{i \in \mathbb{N_+}} L^i$ der  $\epsilon$-freie Konkatenationsabschluss von $F$
	\end{itemize}
	\end{KITexampleblock}\\
\ \\
Selbes wie $L^*$, nur ohne $L^0= \{ \epsilon \}$
\end{frame}



\end{document}