% !TEX TS-program = pdflatex
% !TEX encoding = UTF-8 Unicode

% This is a simple template for a LaTeX document using the "article" class.
% See "book", "report", "letter" for other types of document.

\documentclass[11pt]{article} % use larger type; default would be 10pt

\usepackage[utf8]{inputenc} % set input encoding (not needed with XeLaTeX)
\usepackage{extarrows}
\usepackage{amsfonts}

%%% Examples of Article customizations
% These packages are optional, depending whether you want the features they provide.
% See the LaTeX Companion or other references for full information.

%%% PAGE DIMENSIONS
\usepackage{geometry} % to change the page dimensions
\geometry{a4paper} % or letterpaper (US) or a5paper or....
% \geometry{margin=2in} % for example, change the margins to 2 inches all round
% \geometry{landscape} % set up the page for landscape
%   read geometry.pdf for detailed page layout information

\usepackage{graphicx} % support the \includegraphics command and options

% \usepackage[parfill]{parskip} % Activate to begin paragraphs with an empty line rather than an indent

%%% PACKAGES
\usepackage{booktabs} % for much better looking tables
\usepackage{array} % for better arrays (eg matrices) in maths
\usepackage{paralist} % very flexible & customisable lists (eg. enumerate/itemize, etc.)
\usepackage{verbatim} % adds environment for commenting out blocks of text & for better verbatim
\usepackage{subfig} % make it possible to include more than one captioned figure/table in a single float
% These packages are all incorporated in the memoir class to one degree or another...

%%% HEADERS & FOOTERS
\usepackage{fancyhdr} % This should be set AFTER setting up the page geometry
\pagestyle{fancy} % options: empty , plain , fancy
\renewcommand{\headrulewidth}{0pt} % customise the layout...
\lhead{}\chead{}\rhead{}
\lfoot{}\cfoot{\thepage}\rfoot{}

%%% SECTION TITLE APPEARANCE
\usepackage{sectsty}
\allsectionsfont{\sffamily\mdseries\upshape} % (See the fntguide.pdf for font help)
% (This matches ConTeXt defaults)

%%% ToC (table of contents) APPEARANCE
\usepackage[nottoc,notlof,notlot]{tocbibind} % Put the bibliography in the ToC
\usepackage[titles,subfigure]{tocloft} % Alter the style of the Table of Contents
\renewcommand{\cftsecfont}{\rmfamily\mdseries\upshape}
\renewcommand{\cftsecpagefont}{\rmfamily\mdseries\upshape} % No bold!

\newcommand{\zz}{\mathrm{Z\kern-.3em\raise-0.5ex\hbox{Z}}} %zu zeigen Zeichen

%%% END Article customizations

%%% The "real" document content comes below...

\title{Aufgaben zur vollständige Induktion}
\author{Max Göckel, Tutorium 17}
\date{Grundbegriffe der Informatik}
%\date{} % Activate to display a given date or no date (if empty),
         % otherwise the current date is printed 

\begin{document}
\maketitle
In Aufgabe 1 wird die Induktion an einem einfach Beispiel gezeigt, Aufgaben 2 und 3 zeigen die Induktion über Summen und Produkte, Aufgabe 4 die Induktion mit einer Fließtextaufgabe und 5 mit einer Folge. Die Aufgaben sind entweder aus dem Tutorium, alten Klausuren oder dem Internet.



\section{Einfache Aufgabe aus dem Tutorium}
$\zz \forall n \in \mathbb{N}_0: n^2 + n$ ist gerade.\\
\ \\
\emph{Induktionsanfang:} Für $i=0$: $0^2+0 = 0$.\\
\ \\
\emph{Induktionsvoraussetzung:} Die Behauptung gelte für ein beliebiges aber festes $n \geq 0$.\\
\ \\
\emph{Induktionsschritt:} ($n \rightarrow n+1$)\\
$(n+1)^2 + (n+1) \Leftrightarrow n^2 +2n+1+n+1 \Leftrightarrow (n^2 +n) + 2n +2$\\
\ \\
Mit der Induktionsvoraussetzung $(n^2 + 1)$ ist gerade und $(2n + 2)$ auch gerade ist die Behauptung wahr.\\

Die Behauptung wurde nach dem Prinzip der vollständigen Induktion $\forall n \geq 0$ bewiesen.
\ \\



\section{Aufgabe aus dem Tutorium, Induktion über Summen}
$\zz (1+2+3+...+n)= \sum \limits_{i=1}^n i = \frac{n(n+1)}{2}, \forall n \geq 1$\\
\ \\
\emph{Induktionsanfang:} Für $i=1$: $\sum \limits_{i=1}^1 i = \frac{1(1+1)}{2}$.\\
\ \\
\emph{Induktionsvoraussetzung:} Die Behauptung gelte für ein beliebiges aber festes $n \geq 0$.\\
\ \\
\emph{Induktionsschritt:} ($n \rightarrow n+1$)\\
$\sum \limits_{i=1}^n i \Leftrightarrow (n+1) + \sum \limits_{i=1}^n i \xLeftrightarrow{\text{I.V.}} (n+1) + \frac{n(n+1)}{2} \Leftrightarrow \frac{2(n+1) + n(n+1)}{2} \Leftrightarrow \frac{(n+1)(n+2)}{2}$.\\

Die Behauptung wurde nach dem Prinzip der vollständigen Induktion $\forall n \geq 1$ bewiesen.



\section{WS15/16, Induktion über Produkte}
$\zz$ $ \forall n \in \mathbb{N}_+ \backslash \{1\}:$ $\prod_{i=2}^n (1 - \frac{1}{i} ) = \frac{1}{n}$\\
\ \\
\emph{Induktionsanfang:} $\prod_{i=2}^2 (1 - \frac{1}{i} ) = 1 - \frac{1}{2} = \frac{1}{2}$\\
\ \\
\emph{Induktionsvoraussetzung:} Die Behauptung gelte für ein beliebiges aber festes $n \in \mathbb{N}_+ \backslash \{1\}$.\\
\ \\
\emph{Induktionsschritt:} ($n \rightarrow n+1$)\\
$\prod_{i=2}^n+1 (1 - \frac{1}{i} ) \Leftrightarrow (1- \frac{1}{n+1}) \cdot \prod_{i=2}^n (1 - \frac{1}{i} ) \xLeftrightarrow{\text{I.V.}} (1- \frac{1}{n+1}) \cdot \frac{1}{n} \Leftrightarrow \frac{n}{n+1} \cdot \frac{1}{n} \Leftrightarrow \frac{1}{n+1}$\\

Die Behauptung wurde nach dem Prinzip der vollständigen Induktion $\forall n \in \mathbb{N}_+ \backslash \{1\}$ bewiesen.



\section{Induktion mit einer Fließtextaufgabe}
$\zz$ Dass man jeden glatten Betrag $>$7\$ mit (imaginären) Scheinen im Wert von 3\$ und 5\$ ohne Rückgeld zahlen kann.\\
Formal heißt das: $\forall x \in \mathbb{N}_{>7}: \exists m,n \in \mathbb{N}_0: x = 3 m + 5 n$. ($m,n$ Anzahl der Scheine)\\
\ \\
\emph{Induktionsanfang mit dreifacher Verkettung:}\\
$x=8:     8=1 \cdot 3 + 1 \cdot 5$\\
$x=9:     9=3 \cdot 3 + 0 \cdot 5$\\
$x=10: 10=0 \cdot 3 + 2 \cdot 5$\\
\ \\
\emph{Induktionsvoraussetzung:} Die Behauptung gelte für ein beliebiges aber festes $x \in \mathbb{N}_{>7}$.\\
\ \\
\emph{Induktionsschritt:} $(x \rightarrow x+3)$\\
$x+3 \xLeftrightarrow{\text{I.V. für x}} (3m + 5n) + 3 \Leftrightarrow 3(m+1) + 5n$.\\

Die Behauptung wurde nach dem Prinzip der vollständigen Induktion $\forall x \in \mathbb{N}_{>7}$ bewiesen.



\section{WS08/09: Induktion mit einer Folge}
Die Zahlenfolge F sei wie folgt rekursiv definiert:\\
$F_0 = 0\\
F_1 = 2\\
\forall n \in \mathbb{N}_0: F_{n+2} = 4F_{n+1} - 4F_{n}$.\\
\ \\
$\zz$ dass $\forall n \in \mathbb{N}_0$ die Zahl $F_n$ durch $2^n$ teilbar ist.\\
Mit anderen Worten: $\forall n \in \mathbb{N}_0: \exists k \in \mathbb{Z}: F_n - k \cdot 2^n$\\
\ \\
\emph{Induktionsanfang mit doppelter Verkettung:}\\
$F_0 = 0 = 0 \cdot 2^0$ und $F_1 = 2 = 1 \cdot 2^1$\\
\ \\
\emph{Induktionsvoraussetzung:}\\
Die Behauptung ist umformuliert gleich $F_{n+1} = k_1 \cdot 2^{n+1}$ und $F_n = k_0 \cdot 2^n$\\
Die Behauptung gelte für ein beliebiges aber festes $n \in \mathbb{N}_0$\\
\ \\
\emph{Induktionsschritt $(F_n \rightarrow F_{n+2})$:}\\
$F_{n+2} \Leftrightarrow  4 F_{n+1} - 4 F_n \xLeftrightarrow{\text{IV}} 4(k_1 \cdot 2^{n+1}) - (k_0 \cdot 2^n) \Leftrightarrow 2k_1 \cdot 2^{n+2} - k_0 \cdot 2^{n+2} \Leftrightarrow (2k_1 - k_0) \cdot 2^{n+2}.$\\
Mit $k_0$ und $k_1$ ist auch $2k_1 - k_0$ in $\mathbb{Z}$.\\


Die Aussage wurde mit dem Prinzip der vollständigen Induktion $\forall n \in \mathbb{N}_0$ bewiesen.






\end{document}
