% !TEX TS-program = pdflatex
% !TEX encoding = UTF-8 Unicode

% This is a simple template for a LaTeX document using the "article" class.
% See "book", "report", "letter" for other types of document.

\documentclass[11pt]{article} % use larger type; default would be 10pt

\usepackage[utf8]{inputenc} % set input encoding (not needed with XeLaTeX)

%%% Examples of Article customizations
% These packages are optional, depending whether you want the features they provide.
% See the LaTeX Companion or other references for full information.

%%% PAGE DIMENSIONS
\usepackage{geometry} % to change the page dimensions
\geometry{a4paper} % or letterpaper (US) or a5paper or....
% \geometry{margin=2in} % for example, change the margins to 2 inches all round
% \geometry{landscape} % set up the page for landscape
%   read geometry.pdf for detailed page layout information

\usepackage{graphicx} % support the \includegraphics command and options

% \usepackage[parfill]{parskip} % Activate to begin paragraphs with an empty line rather than an indent

%%% PACKAGES
\usepackage{booktabs} % for much better looking tables
\usepackage{array} % for better arrays (eg matrices) in maths
\usepackage{paralist} % very flexible & customisable lists (eg. enumerate/itemize, etc.)
\usepackage{verbatim} % adds environment for commenting out blocks of text & for better verbatim
\usepackage{subfig} % make it possible to include more than one captioned figure/table in a single float
% These packages are all incorporated in the memoir class to one degree or another...

%%% HEADERS & FOOTERS
\usepackage{fancyhdr} % This should be set AFTER setting up the page geometry
\pagestyle{fancy} % options: empty , plain , fancy
\renewcommand{\headrulewidth}{0pt} % customise the layout...
\lhead{}\chead{}\rhead{}
\lfoot{}\cfoot{\thepage}\rfoot{}

%%% SECTION TITLE APPEARANCE
\usepackage{sectsty}
\allsectionsfont{\sffamily\mdseries\upshape} % (See the fntguide.pdf for font help)
% (This matches ConTeXt defaults)

%%% ToC (table of contents) APPEARANCE
\usepackage[nottoc,notlof,notlot]{tocbibind} % Put the bibliography in the ToC
\usepackage[titles,subfigure]{tocloft} % Alter the style of the Table of Contents
\renewcommand{\cftsecfont}{\rmfamily\mdseries\upshape}
\renewcommand{\cftsecpagefont}{\rmfamily\mdseries\upshape} % No bold!

%%% END Article customizations

%%% The "real" document content comes below...

\title{Brief Article}
\author{The Author}
%\date{} % Activate to display a given date or no date (if empty),
         % otherwise the current date is printed 

\begin{document}
\maketitle

\section{Schreibe ein Programm dass für ein x $\geq$ 0 an Adresse adr \emph{x div 2}
ausführt und und das Ergebnis wieder in adr abspeichert }

LDC 1 -- lädt 000000000000000000000001\\
NOT -- macht aus 00....001 eine 11111111111111111111111110\\
AND adr -- 1 genau dann wenn beide 1 ist, also wird letztes Zeichen in adr 0\\
RAR -- 1 nach rechts ist das selbe wie div 2, letztes Zeichen (0) wird zu erstem Zeichen\\
STV adr -- speichern\\
HALT -- MIMA anhalten\\

\section{Schreibe ein Programm, das für den Wert x an der Adresse 001 \emph{x mod
2} berechnet und das Ergebnis wieder in x speichert}

LDC 1 -- 000000000000000000000001 laden\\
AND 001 -- 00...01 mit mit Wert an Adresse 001 verODERn\\
STV 001 -- speichern\\
HALT\\

\section{Schreibe ein Programm, dass den Wert an adr2 mit x multipliziert}
Wobei x and adr1 steht, adr1-10 stehen zur Verfügung.\\
\ \\
 LDC 0 \\
NOT \\
STV adr3 -- -1 an adr3 \\
LDV adr2 -- Wert von adr2 abspeichern \\
STV adr4 \\
loop: LDV adr1 \\
ADD adr3 \\
STV adr1 -- adr1 - 1 rechnen und speichern \\
JMN end \\
LDV adr2 \\
ADD adr4 \\
STV adr2 -- adr2 + adr4 rechnen und speichern \\
JMP loop \\
end: HALT
\end{document}
