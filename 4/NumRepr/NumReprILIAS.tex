% !TEX TS-program = pdflatex
% !TEX encoding = UTF-8 Unicode

% This is a simple template for a LaTeX document using the "article" class.
% See "book", "report", "letter" for other types of document.

\documentclass[11pt]{article} % use larger type; default would be 10pt

\usepackage[utf8]{inputenc} % set input encoding (not needed with XeLaTeX)

%%% Examples of Article customizations
% These packages are optional, depending whether you want the features they provide.
% See the LaTeX Companion or other references for full information.

%%% PAGE DIMENSIONS
\usepackage{geometry} % to change the page dimensions
\geometry{a4paper} % or letterpaper (US) or a5paper or....
\geometry{margin=0.83in} % for example, change the margins to 2 inches all round
% \geometry{landscape} % set up the page for landscape
%   read geometry.pdf for detailed page layout information

\usepackage{graphicx} % support the \includegraphics command and options

% \usepackage[parfill]{parskip} % Activate to begin paragraphs with an empty line rather than an indent

%%% PACKAGES
\usepackage{booktabs} % for much better looking tables
\usepackage{array} % for better arrays (eg matrices) in maths
\usepackage{paralist} % very flexible & customisable lists (eg. enumerate/itemize, etc.)
\usepackage{verbatim} % adds environment for commenting out blocks of text & for better verbatim
\usepackage{subfig} % make it possible to include more than one captioned figure/table in a single float
\usepackage{amsmath}
% These packages are all incorporated in the memoir class to one degree or another...

%%% HEADERS & FOOTERS
\usepackage{fancyhdr} % This should be set AFTER setting up the page geometry
\pagestyle{fancy} % options: empty , plain , fancy
\renewcommand{\headrulewidth}{0pt} % customise the layout...
\lhead{}\chead{}\rhead{}
\lfoot{}\cfoot{\thepage}\rfoot{}

%%% SECTION TITLE APPEARANCE
\usepackage{sectsty}
\allsectionsfont{\sffamily\mdseries\upshape} % (See the fntguide.pdf for font help)
% (This matches ConTeXt defaults)

%%% ToC (table of contents) APPEARANCE
\usepackage[nottoc,notlof,notlot]{tocbibind} % Put the bibliography in the ToC
\usepackage[titles,subfigure]{tocloft} % Alter the style of the Table of Contents
\renewcommand{\cftsecfont}{\rmfamily\mdseries\upshape}
\renewcommand{\cftsecpagefont}{\rmfamily\mdseries\upshape} % No bold!

%%% END Article customizations

%%% The "real" document content comes below...

\title{$Num_k$ und $Repr_k$}
\author{Tutorium 17, \#4}
\date{} % Activate to display a given date or no date (if empty),
         % otherwise the current date is printed 

\begin{document}
\maketitle
\ \\
\section{$Num_k$}%%%%%%%%%%%%%%%%%%%%%%%%%%%%%%%

Ist w ein Wort, welches eine Zahl zur Basis k darstellt (z.B. $FFF_{16}$ oder $1001_2$), so ist $Num_k(w)$ die Darstellung des Wortes / der Zahl im Dezimalsystem (Basis 10).

\subsection{Definition} %%%%%%%

Sei $w = w' \cdot x$.
\begin{itemize}
	\item $Num_k(\epsilon) = 0$
	\item $Num_k(w' \cdot x) = k \cdot Num_k(w') + num_k(x)$, 
	\begin{itemize}
		\item[(!)] "$+$" steht hier für die Addition, nicht für irgendeine Konkatenation.
	\end{itemize}
	\item $num_k(x) = x$
\end{itemize}



\subsection{Beispiel} %%%%%%%%%%

Sei $w_1 = 5_8$ (5 zur Basis 8).\\
$Num_8(5)\\
= 8 \cdot Num_8(\epsilon) + num_8(5)\\
= 8 \cdot 0 + 5\\
= 5$.\\
\ \\
Sei $w_2 = 234_9$ (234 zur Basis 9).\\
$Num_9(234) \\
= 9 \cdot Num_9(23) + num_9(4) \\
= 9 \cdot ((9 \cdot Num_9(\epsilon 2)) + num_9(3)) + num_9(4) \\
= 9 \cdot ( 9 \cdot (9 \cdot Num_9(\epsilon)) + num_9(2)) + num_9(3) ) + num_9(4) \\
= 9 \cdot ((9 \cdot 2) + 3) + 4 \\
= (9 \cdot 9 \cdot 2) + (9 \cdot 3) + 4 \\
= 162 + 27 + 4\\
= 193$. \\
\ \\
Sei $w_3 = B66_{16}$ (In Hexadezimal gilt B = 11, also  $w = 11 \cdot 6 \cdot 6$ in der Basis $16$).\\
$Num_{16}(B66) \\
= 16 \cdot Num_{16}(B6) + num_{16}(6) \\
= 16 \cdot (16 \cdot Num_{16}(\epsilon B) + num_{16}(6)) + num_{16}(6) \\
= 16 \cdot (16 \cdot (16 \cdot Num_{16}(\epsilon) num_{16}(B)) + num_{16}(6)) + num_{16}(6) \\
= 16 \cdot (16 \cdot 11 + 6) + 6 \\
= (16 \cdot 16 \cdot 11) + (16 \cdot 6) + 6 \\
= 2816 + 96 + 6\\
= 2918$ \\

\section{$Repr_k$} %%%%%%%%%%%%%%%%%%%%%%%%%%%%%%%%%%%%%
Während $Num_k$ von anderen Systemen in Dezimal umwandelt, ist $Repr_k$ dafür da, eine ein Wort w welches eine Zahl in dezimal darstellt in eine Zahl vom System k umzuwandeln. Um also von Hexadezimal auf binär zu kommen ist $Repr_2(Num_{16}(w))$, die Hintereinanderauführung von $Num_{k_1}$ und $Repr_{k_2}$, nötig.

\subsection{Definition} %%%%%%%%%%%
Sei $w = x_{10}$ eine Dezimalzahl die in Basis k umgerechnet werden soll.
\begin{itemize}
	\item Fall $x < k: repr_k(x) = x$
	\item Fall $x \geq k: Repr_k($x div k$) \cdot repr_k(n \bmod k)$
	\begin{itemize}
		\item[(!)] "$\cdot$" steht hier wieder für die Konkatenation
		\item[(!)] In den Beispielen wird anstelle von "x div k" die Darstellung $\frac{x}{k}$ genutzt. Hier soll es das selbe beschreiben, nämlich das Teilen mit Rest ($\frac{74}{10} = 7, \frac{36}{7} = 5$), in der Klausur und auf den ÜBs bitte nicht so verwenden.
	\end{itemize}
\end{itemize}


\subsection{Beispiele} %%%%%%%%%%%
$w = 29$ und $k = 3$.\\
$ Repr_3(29) \\
= Repr_3(\frac{29}{3}) \cdot repr_3(29 \bmod 3) \\
= Repr_3(9) \cdot 2 \\
= Repr_3(\frac{9}{3}) \cdot repr_3(9 \bmod 3) \cdot 2 \\
= Repr_3(3)\cdot 0 \cdot 2 \\
= Repr_3(1) \cdot repr_3(0) \cdot 0 \cdot 2 \\
= repr_3(1) \cdot 0 \cdot 0 \cdot 2 \\
= 1 \cdot 0 \cdot 0 \cdot 2 = 1002_3\\$
\ \\
$w = 53$ und $k = 5$.\\
$ Repr_5(53) \\
= Repr_5(\frac{53}{5}) \cdot repr_5(53 \bmod 5) \\
= Repr_5(10) \cdot repr_5(3)\\
= Repr_5(\frac{10}{5}) \cdot repr_5(10 \bmod 5) \cdot 3\\
= Repr_5(2) \cdot 0 \cdot 3 \\
= 2 \cdot 0 \cdot 3 = 203_5$




















\end{document}
