\documentclass[ngerman,hyperref={pdfpagelabels=false}]{beamer}

% -----------------------------------------------------------------------------

\graphicspath{{images/}}

% -----------------------------------------------------------------------------

\usetheme{KIT}

\setbeamercovered{transparent}
%\setbeamertemplate{enumerate items}[ball]

\newenvironment<>{KITtestblock}[2][]
{\begin{KITcolblock}<#1>{#2}{KITblack15}{KITblack50}}
{\end{KITcolblock}}

\usepackage[ngerman,english]{babel}
\usepackage[utf8]{inputenc}
\usepackage[TS1,T1]{fontenc}
\usepackage{array}
\usepackage{multicol}
\usepackage[absolute,overlay]{textpos}
\usepackage{beamerKITdefs}
\usepackage{amsfonts}


\pdfpageattr {/Group << /S /Transparency /I true /CS /DeviceRGB>>}	%required to prevent color shifting withd transparent images


\title{Tutorium 17, \#4}
\subtitle{Max Göckel-- \textit{uzkns@student.kit.edu}}

\author[Max Göckel]{Max Göckel}
\institute{Institut für Theoretische Informatik - Grundbegriffe der Informatik}

\TitleImage[width=\titleimagewd,height=\titleimageht]{titel}

\KITinstitute{Institut f\"ur Theoretische Informatik}
\KITfaculty{Fakult\"at f\"ur Informatik}

% -----------------------------------------------------------------------------

\begin{document}
\setlength\textheight{7cm} %required for correct vertical alignment, if [t] is not used as documentclass parameter


% title frame
\begin{frame}
  \maketitle
\end{frame}


%Num_k
\begin{frame}
\frametitle{$Num_k$}
Wenn w ein Wort ist, das eine Zahl zur Basis k darstellt, so ist $Num_k(w)$ die Darstellung dieses Wortes/dieser Zahl im Dezimalsystem.\\
\ \\
\begin{itemize}
	\item Definition der Funktion mit w=w'$\cdot$x.
	\begin{itemize}
		\item $Num_k(\epsilon)=0$
		\item $Num_k(w'x) = k \cdot Num_k(w') + num_k(x)$ mit $x \in \mathbb{Z}_k$ und $w' \in \mathbb{Z}_k^*$
		\item $num_k(x) = x$
	\end{itemize}
\end{itemize}
\end{frame}


%Bsp 1
\begin{frame}
\frametitle{$Num_k$: Beispiel}
Sei $w_1 = 5_8$ (5 zur Basis 8).\\
\ \\
$Num_8(5)$\\
$= 8 \cdot Num_8(\epsilon) + num_8(5)$\\
$= 8 \cdot 0 + 5$\\
$= 5$.
\end{frame}


%Lösung
\begin{frame}
\frametitle{$Num_k$: Aufgabe}

Sei $w_3 = B66_{16}$ (In Hexadezimal gilt B = 11, also  $w = 11 \cdot 6 \cdot 6$ in der Basis $16$).\\
\ \\
$Num_{16}(B66)$ \\
$= 16 \cdot Num_{16}(B6) + num_{16}(6) $\\
$= 16 \cdot (16 \cdot Num_{16}(\epsilon B) + num_{16}(6)) + num_{16}(6)$ \\
$= 16 \cdot (16 \cdot (16 \cdot Num_{16}(\epsilon) num_{16}(B)) + num_{16}(6)) + num_{16}(6) $\\
$= 16 \cdot (16 \cdot 11 + 6) + 6 $\\
$= (16 \cdot 16 \cdot 11) + (16 \cdot 6) + 6 $\\
$= 2816 + 96 + 6$\\
$= 2918$ \\
\end{frame}


%Repr_k
\begin{frame}
\frametitle{$Repr_k$}
$Repr_k$ ist die Darstellung einer Dezimalzahl in der Basis $k$.
\ \\
\begin{itemize}
	\item Definition:
	\begin{itemize}
		\item n < k: $repr_k(n)$
		\item n $\geq$ k: $Repr_k(n$ div $k) \cdot repr_k(n$ mod $k)$
	\end{itemize}  
\end{itemize} 
\end{frame}


%Repr Beispiel
\begin{frame}
\frametitle{$Repr_k$: Beispiel}

$w = 29$ und $k = 3$.\\
\ \\
$ Repr_3(29)$ \\
$= Repr_3(\frac{29}{3}) \cdot repr_3(29 \bmod 3)$ \\
$= Repr_3(9) \cdot 2$ \\
$= Repr_3(\frac{9}{3}) \cdot repr_3(9 \bmod 3) \cdot 2$ \\
$= Repr_3(3)\cdot 0 \cdot 2$ \\
$= Repr_3(1) \cdot repr_3(0) \cdot 0 \cdot 2$ \\
$= repr_3(1) \cdot 0 \cdot 0 \cdot 2$ \\
$= 1 \cdot 0 \cdot 0 \cdot 2 = 1002_3$\\
\ \\
\end{frame}


%LÖSUNG
\begin{frame}
\frametitle{$Repr_k$: Aufgabe}
$w = 53$ und $k = 5$.\\
\ \\
$ Repr_5(53)$ \\
$= Repr_5(\frac{53}{5}) \cdot repr_5(53 \bmod 5)$ \\
$= Repr_5(10) \cdot repr_5(3)$ \\
$= Repr_5(\frac{10}{5}) \cdot repr_5(10 \bmod 5) \cdot 3$ \\
$= Repr_5(2) \cdot 0 \cdot 3$ \\
$= 2 \cdot 0 \cdot 3 = 203_5$
\end{frame}


\begin{frame}
	\frametitle{Binärsystem}
Im Dezimalsystem gibt es die Ziffern $1..9$. \\
\ \\
Im Binär- oder Dualsystem gibt es nur 2 Ziffern: $0, 1$.\\
Die Wertigkeit jeder Ziffer $Z$ ist dabei $Z \cdot 2^i$.


\end{frame}


\begin{frame}
\frametitle{Binärtabelle}

\begin{tabular}{c|c|c|c|c}
	Wertigkeit: & $2^3=8$ & $2^2=4$ & $2^1=2$ & $2^0=1$ \\
	\hline
	0  & 0 & 0 & 0 & 0 \\
	1  & 0 & 0 & 0 & 1 \\
	2  & 0 & 0 & 1 & 0 \\
	3  & 0 & 0 & 1 & 1 \\
	4  & 0 & 1 & 0 & 0 \\
	5  & 0 & 1 & 0 & 1 \\
	6  & 0 & 1 & 1 & 0 \\
	7  & 0 & 1 & 1 & 1 \\
	8  & 1 & 0 & 0 & 0 \\
	9  & 1 & 0 & 0 & 1 \\
	10 & 1 & 0 & 1 & 0 \\
	11 & 1 & 0 & 1 & 1 \\
	12 & 1 & 1 & 0 & 0 \\
\end{tabular}
\ \\
 \ \\
Zahl 1100: $1 \cdot 2^3 + 1 \cdot 2^2 + 0 \cdot 2^1 + 0 \cdot 2^0 = 8 + 4 + 0 + 0 = 12$ 

\end{frame}



%Zweierkomplement
\begin{frame}
  \frametitle{Zweierkomplement}
Zweierkomplementdarstellung ($\mathbb{K}_n$) ist die übliche Darstellung von ganzen Zahlen im PC. Dabei "opfern" wir das erste Binär-Bit für das Vorzeichen.\\
Zahlen werden gleichmäßig auf den positiven und negativen Bereich verteilt. Mit der 0 ist im positiven Bereich eine Zahl weniger als im negativen.\\
\ \\
\begin{KITexampleblock}{Definition}
	\begin{itemize}
		\item $\mathbb{K}_n = \{x \in \mathbb{Z} | -2^{n-1} \leq x\geq 2^{n-1} -1\}$
	\end{itemize}
\end{KITexampleblock}

\ \\
  \begin{itemize}
	\item $\mathbb{K}_7 = \{-64, -63, ..., ..., 62, 63\}$ (Alle Zahlen bis $2^6$, im positiven eine Zahl weniger)
  \end{itemize} 
\end{frame}


%Umrechung Zkpl
\begin{frame}
  \frametitle{$Zkpl_k$}
Umrechung eriner n-stelligen Binärzahl ins Zweierkomplement mit $Zkpl_n$.\\
\ \\
Vorgehen bei negativen Zahlen:
  \begin{itemize}
	\item Nullen durch Einsen erstzen und Einsen durch Nullen (Invertieren)
	\item 1 dazu addieren
	\item Schauen ob die Zahl vorher positiv oder negativ war, entsprechend das erste Bit anpassen
  \end{itemize}
\ \\
Bei positiven Zahlen:
  \begin{itemize}
	\item Nichts tun, evtl. auf richtige Länge mit 0'len auffüllen
  \end{itemize}
\end{frame}


%Umrechung Zkpl Beispiel
\begin{frame}
  \frametitle{$Zkpl_k$: Beispiel}
Beispielzahl ist ${-01101011}_2$\\
\ \\
Vorgehen:
  \begin{itemize}
	\item Invertieren: 10010100
	\item +1: 10010101
  \end{itemize}
\end{frame}


%LÖSUNG
\begin{frame}
  \frametitle{$Zkpl_k$: Aufgabe}
Rechnet die fogenden Zahlen mit Zwischenschritten um:\\
\ \\
  \begin{itemize}
	\item $5 = {101}_2 = {0101}_{ZK}$
	\item ${-100101}_2 = {111011011}_{ZK}$
	\item ${11110}_{ZK} = {-10}_2$
  \end{itemize}
\end{frame}


%Homos
\begin{frame}
  \frametitle{Homomorphismen}
A,B so bildet eine Abbildung $h: A \rightarrow B^*$ einen Buchstaben von A uf ein Wort aus B ab.\\
ein Homorphismus $h^{**}: A^* \rightarrow B^*$ macht das selbe mit einem ganzen Wort aus A, wobei jedes Zeichen einzeln abgebildet wird.\\
Ist ein Homomorphismus präfixfrei so existiert eine Umkehrfunktion die aus dem Wort aus B* wider das Wort in A* macht\\
\ \\
\begin{KITexampleblock}{Definition}
	\begin{itemize}
		\item $\forall w \in A^*: \forall x \in A: h^{**}(wx) = h^{**}(w) \cdot h(x)$
	\end{itemize}
	\end{KITexampleblock}
\end{frame}


%Beispiel
\begin{frame}
  \frametitle{Homomorphismen: Beispiel}
Sei $h(a)=101, H(b)=0, h(c)=1$\\
\ \\
  \begin{itemize}
	\item $h^{**}(a) = h(a) = 101$
	\item $h^{**}(cbc) = h(c) \cdot h(b) \cdot h(c) = 101$
  \end{itemize}
\end{frame}


%Huffmann-Codierung
\begin{frame}
  \frametitle{Huffman-Codierung}
Die Huffman-Codierung ist ein präfixfreier Homomorphismus $A^* \rightarrow {\mathbb{Z}_2}^*$, der häufigen Zeichen eine möglichst kurze Abbildung gibt.\\
\ \\
Sei $N_w(x)$ die Anzahl der Vorkommnisse des Buchstaben x in w und "$N_w(x), x$" die Beschriftung des Blattes in einem Baum

  \begin{itemize}
	\item Verbinde die zwei "kleinsten" Werte
	\item Schreibe $N_w(x_1) + N_w(x_2)$ in den neuen Knoten
	\item So lange wiederholen, bis der Baum komplett ist
  \end{itemize}
\end{frame}


%Beispiel
\begin{frame}
  \frametitle{Huffman: Beispiel}
w = bcccabdd, so sind $N_w(a) = 1, N_w(b) = 2, N_w(c) = 3, N_w(d) = 2$ die Häufigkeiten in w.
\end{frame}


%Aufgabe
\begin{frame}
  \frametitle{Huffman: Aufgabe}
Stelle das folgende Wort in einem Huffman-Baum dar und erstelle das Codewort:\\
\ \\
w = ededddbcfbedaccb\\
\end{frame}


%Aufgabe
\begin{frame}
  \frametitle{Huffman: Frage}
Ist ein Huffman-Baum eindeutig?\\
  \begin{itemize}
	\item Nein, da bei mehreren gleich-häufig vorkommenden Wörern man sich aussuchen kann, welche Kante man zuerst verbindet.
  \end{itemize}
\ \\
\ \\
Kann man ein Huffman-Wort wieder decodieren?\\
  \begin{itemize}
	\item Ja, dafür benötigt man aber den Baum.
  \end{itemize}
\end{frame}


%Huffmann-Sonderfall
\begin{frame}
  \frametitle{Huffman-Codierung}
Kommen viele Teilwörter in einem Wort w häufig vor, so macht es Sinn die Codierung mit kurzen Wörtern anstelle von einzelnen Buchtaben durchzuführen.\\
Ansonsten bleibt das Vorgehen gleich, die Suche nach den besten Teilwörter n kann aber etwas Zeit in Anspruch nehmen.\\
\ \\
Beispielwort: $w = abcabccbdabc = abc \cdot abc \cdot cbd \cdot abc$
  \begin{itemize}
	\item h(abc) = 0, h(cbd) = 1
	\item $h^{**}(w) = 0010$
	\item $|w| = 12, |h^{**}(w)| = 4$, Platzeinsparung um den Faktor 3
  \end{itemize}
\end{frame}


\end{document}
