\documentclass[ngerman,hyperref={pdfpagelabels=false}]{beamer}

% -----------------------------------------------------------------------------

\graphicspath{{images/}}

% -----------------------------------------------------------------------------

\usetheme{KIT}

\setbeamercovered{transparent}
%\setbeamertemplate{enumerate items}[ball]

\newenvironment<>{KITtestblock}[2][]
{\begin{KITcolblock}<#1>{#2}{KITblack15}{KITblack50}}
{\end{KITcolblock}}

\usepackage[ngerman,english]{babel}
\usepackage[utf8]{inputenc}
\usepackage[TS1,T1]{fontenc}
\usepackage{array}
\usepackage{multicol}
\usepackage[absolute,overlay]{textpos}
\usepackage{beamerKITdefs}
\usepackage{amsfonts}

\newcommand{\code}[1]{\texttt{#1}}


\pdfpageattr {/Group << /S /Transparency /I true /CS /DeviceRGB>>}	%required to prevent color shifting withd transparent images


\title{Tutorium 42, \#10}
\subtitle{Max Göckel-- \textit{uzkns@student.kit.edu}}

\author[Max Göckel]{Max Göckel}
\institute{Institut für Theoretische Informatik - Grundbegriffe der Informatik}

\TitleImage[width=\titleimagewd,height=\titleimageht]{titel}

\KITinstitute{Institut f\"ur Theoretische Informatik}
\KITfaculty{Fakult\"at f\"ur Informatik}

% -----------------------------------------------------------------------------

\begin{document}
\setlength\textheight{7cm} %required for correct vertical alignment, if [t] is not used as documentclass parameter


% title frame
\begin{frame}
  \maketitle
\end{frame}


%Adjanzenzmatrix
\begin{frame}
  \frametitle{Adjazenzmatrizen}
Man kann eine Graph nicht nur als Zeichnung darstellen, sondern auch als Matrix.\\
Diese Matrizen heißen Adjazenzmatirzen, sie sind immer quadratisch und gehören eindeutig zu einem Graphen.\\
\ \\
An einer Adjazenzmatrix kann man die Kanten eines Graphen ablesen.
\end{frame}


%Aufbau Matrix
\begin{frame}
  \frametitle{Adjazenzmatrizen: Aufbau}
\begin{itemize}
	\item Jede Zeile i steht für einen Knoten $k_i \in V$
	\item Jede Spalte j steht für die von $k_i$ direkt erreichbaren Knoten
	\begin{itemize}
		\item $ 1 \Leftrightarrow \exists \{ k_i , k_k \} \in E $, es existiert eine Kante zwischen den beiden Knoten
		\item $ 0 \Leftrightarrow \neg \exists \{ k_i , k_k \} \in E $, es existiert keine Kante
	\end{itemize}
\end{itemize}
\end{frame}


%Beispiel
\begin{frame}
  \frametitle{Adjazenzmatrizen: Beispiel}

Die folgende Beispielmatrix beschreibt einen Graphen mit 4 Knoten:\\
\ \\

$A = 
\begin{pmatrix}
1 & 1 & 1 & 0 \\
0 & 0 & 1 & 1 \\
0 & 1 & 1 & 0 \\
0 & 0 & 0 & 0
\end{pmatrix}
$ 
\end{frame}


%%Beispiel
\begin{frame}
  \frametitle{Adjazenzmatrizen: Ungerichtete Graphen}
Bei ungerichteten Graphen ändert sich die Adjazenzmatrix nicht, die einzige Besonderheit ist dass sie symmetrisch ist.\\
\ \\
Die folgende Beispielmatrix beschreibt den selben Graphen wie eben als ungerichteten Graph: \\
\ \\

$B = 
\begin{pmatrix}
1 & 1 & 1 & 0 \\
1 & 0 & 1 & 1 \\
1 & 1 & 1 & 0 \\
0 & 1 & 0 & 0
\end{pmatrix}
$ 
\end{frame}


%Aufgabe
\begin{frame}
  \frametitle{Adjazenzmatrizen: Aufgaben}
Bestimme den Graph anhand der Adjazenzmatrix:\\
$ X =
\begin{pmatrix}
1 & 1 & 1 & 0 & 0 \\
1 & 0 & 1 & 1 & 0 \\
1 & 1 & 0 & 1 & 1 \\
0 & 1 & 1 & 1 & 0 \\
0 & 0 & 1 & 1 & 0
\end{pmatrix}
$ 
\end{frame}

%LÖSUNG
\begin{frame}
  \frametitle{Adjazenzmatrizen: Lösung}
$ A_C =
\begin{pmatrix}
1 & 1 & 1 & 0 & 0 \\
1 & 0 & 1 & 1 & 0 \\
1 & 1 & 0 & 1 & 1 \\
0 & 1 & 1 & 0 & 1 \\
0 & 0 & 1 & 1 & 0
\end{pmatrix}
$ \\
\ \\
$ A_D =
\begin{pmatrix}
0 & 0 & 1 & 0 & 0 & 0 \\
0 & 0 & 0 & 1 & 0 & 0 \\
0 & 0 & 0 & 0 & 1 & 0 \\
0 & 0 & 0 & 0 & 0 & 1 \\
1 & 0 & 0 & 0 & 0 & 0 \\
0 & 1 & 0 & 0 & 0 & 0 
\end{pmatrix}
$
\end{frame}


%Pfade bel. Länge
\begin{frame}
  \frametitle{Adjazenzmatrizen: Pfade von Länge 2}
Eine Adjazenzmatrix $A$ zeigt an, ob es einen direkten Pfad von $k_i$ zu $k_j$ gibt. \\
Um nun zu gucken ob es einen Pfad der Länge 2 gibt (von $k_i$ über $k_k$ zu $k_j$) müssen wir $A^2$ errechnen.
\end{frame}


%Matrixmultiplikation
\begin{frame}
  \frametitle{Adjazenzmatrizen: Pfade von Länge 2}
Eine Adjazenzmatrix $A$ zeigt an, ob es einen direkten Pfad von $k_i$ zu $k_j$ gibt. \\
Um nun zu gucken ob es einen Pfad der Länge 2 gibt (von $k_i$ über $k_k$ zu $k_j$) müssen wir $A^2$ errechnen.\\
\ \\
Matrixmultiplikation:\\
$ A =
\begin{pmatrix}
1 & 2 & 3 \\
3 & 1 & 2 \\
2 & 1 & 4 \\
\end{pmatrix}
$ wird dann zu 
$ A^2 =
\begin{pmatrix}
13 & 7 & 19 \\
10 & 9 & 19 \\
13 & 9 & 24 \\
\end{pmatrix}
$ \\
\ \\
Element $a^2_{33} = a_{31} \cdot a_{13} + a_{32} \cdot a_{23} + a_{33} \cdot a_{33} = 2 \cdot 3 + 1 \cdot 2 + 4 \cdot 4 = 6 + 2 + 16 = 24$\\
\ \\
Für Pfade Länge x berechnet man dann einfach $A^x$.
\end{frame}


%Beispiel
\begin{frame}
  \frametitle{Adjazenzmatrizen: Beispiel}
$ A =
\begin{pmatrix}
1 & 1 & 0 & 0 \\
1 & 0 & 0 & 0 \\
0 & 1 & 0 & 0 \\
1 & 0 & 1 & 0
\end{pmatrix}
\cdot
\begin{pmatrix}
1 & 1 & 0 & 0 \\
1 & 0 & 0 & 0 \\
0 & 1 & 0 & 0 \\
1 & 0 & 1 & 0
\end{pmatrix}
=
\begin{pmatrix}
2 & 1 & 0 & 0 \\
1 & 1 & 0 & 0 \\
1 & 0 & 0 & 0 \\
1 & 2 & 0 & 0
\end{pmatrix}
$\\
\ \\
Die $2$ bedeutet es gibt 2 Wege von Knoten 4 zu 2.
\end{frame}


%Wegematrix
\begin{frame}
  \frametitle{Wegematrizen}
Eine x-mal mit sich selbst multiplizierte Adjazenzmatrix zeigt uns einen Pfad der Länge x an.\\
Was ist nun wenn wir wissen wollen ob \emph{überhaupt} ein Pfad von A nach B existiert (in egal wie vielen Schritten).\\
\ \\
Dazu brauchen wir die Wegematrix, eine Adjazenzmatrix, die für einen Pfad bel. Länge von A zu B eine 1 hat und sonst eine 0.\\
\end{frame}


%LÖSUNG
\begin{frame}
  \frametitle{Wegematrizen: Aufgabe}
Bestimme zu Graph E eine 4x4-Wegematrix.\\
\ \\
$ A_E =
\begin{pmatrix}
1 & 1 & 1 & 0 \\
0 & 1 & 1 & 0 \\
0 & 0 & 1 & 0 \\
0 & 1 & 1 & 1
\end{pmatrix}
$
\end{frame}


%Warshall Algorithmus
\begin{frame}
  \frametitle{Warshallalgorithmus}
Um eine Wegematrix zu errechnen hilft der Warshall-Algorithmus.\\
\ \\
Dieser errechnet in n Durchgängen (n ist dabei die Anzahl an Knoten in einem Graph) aus einer Adjazenzmatrix eine Wegematrix.
\end{frame}


%Warshall Algorithmus
\begin{frame}
  \frametitle{Warshallalgorithmus: Aufbau}
\begin{itemize}
	\item In der Adjazenzmatrix alle Diagonalwerte auf 1 setzen
	\item n Durchgänge: Im $i$-ten Durchgang alle Nullen in der Matrix angucken
	\begin{itemize}
		\item 0 durch 1 ersetzten, wenn in der Zeile und Spalte an der i-ten Stelle je eine 1 steht
	\end{itemize}
\end{itemize}
\end{frame}


%Beispiel
\begin{frame}
  \frametitle{Warshallalgorithmus: Beispiel}
Ein kurzes Beispie aus Durchgang 2 ($i = 2$):\\
Wir betrachten $a_{33} = 0$
\ \\
$
\begin{pmatrix}
- & - & 1 & - \\
- & - & 1 & - \\
0 & 1 & 0 & 0 \\
- & - & 1 & -
\end{pmatrix}
$\\
\ \\
Da $a_{33}$ in der 3. Spalte und Zeile steht und $i = 2$ nehmen wir von den Zeilen/Spalten das zweite Element ($a_{32}$ und $a_{23}$).\\
Diese sind beide $1$, so wird $a_{33}$ im nächsten Durchgang auch zur 1.
\end{frame}


%LÖSUNG
\begin{frame}
  \frametitle{Warshallalgorithmus: Lösung}
Bestimme zu den Graphen $F, G$ die Wegematrix. \\
$
\begin{pmatrix}
1 & 1 & 1 & 1 \\
1 & 1 & 1 & 1 \\
1 & 1 & 1 & 1 \\
1 & 1 & 1 & 1
\end{pmatrix}
$ \\
\ \\
$
\begin{pmatrix}
1 & 1 & 1 & 1 \\
0 & 1 & 1 & 1 \\
0 & 0 & 1 & 1 \\
0 & 0 & 0 & 1
\end{pmatrix}
$
\end{frame}


%Zeitkomplexität
\begin{frame}
  \frametitle{Algorithmen: Zeitkomplexität}
Wir kennen (aus Programmieren) schon mehrere Sortieralgorithmen:\\
\begin{itemize}
	\item BubbleSort
	\item InsertionSort
	\item BogoSort / RandomSort
	\item MergeSort
\end{itemize}
\ \\
Wieso gibt es mehrere Sortieralgorithmen wenn doch einer ausreichen würde um Arrays zu sortieren?
\end{frame}


%Vergleiche
\begin{frame}
  \frametitle{Algorithmen: Vergleiche}
Verschiedene Algorithmen haben unterschiedliche Laufzeiten \emph{("Zeitkomplexität")}.\\
So ist MergeSort im worst-case immer noch schneller als InsertionSort im worst-case.\\
\ \\
Um die Zeitkomplexität von Algorithmen anzugeben interessiert uns immer der worst-case. Wir probieren eine obere Schranke zu finden und zwar nicht in Sekunden/Millisekunden, sondern in Abhängigkeit von $n$.\\
Konstante Faktoren und Konstanten werden dabei irgnoriert, d.h. $1000 n + 10$ und $0,5 n$ haben die gleiche Komplexität, nämlich $n$.
\end{frame}


%Größenordnung
\begin{frame}
  \frametitle{Algorithmen: Laufzeiten}
Die häufigsten unterschiedlichen Größenordnungen sind (aufsteigend sortiert): \\
\begin{itemize}
	\item $1$ (atomare Ausführung)
	\item $log(n)$
	\item $n$
	\item $n \cdot log(n)$
	\item $n^2$
	\item $n^3$
	\item $2^n$
	\item $n!$
	\item $n^n$
\end{itemize}
\end{frame}


%Landau Notation
\begin{frame}
  \frametitle{Algorithmen: O-Kalkül}
Um Algorithmen zu vergleichen und einzuordnen gibt es verschiedene Schranken: \\
\begin{itemize}
	\item Obere Schranke $O$ (Groß-O): In $O( f(n) )$ sind alle Funktionen die langsamer oder gleich schnell wie f wachsen.
	\item Untere Schranke $\Omega$ (Omega): In $\Omega ( f(n) )$ sind alle Funktionen die schneller oder gleich schnell wie f wachsen.
	\item Obere u. untere Schranke $\Theta$ (Theta): In $\Theta ( f(n) )$ sind alle Funktionen die genau gleich schnell wie f wachsen.
\end{itemize}
\ \\
Bei den Vergleichen schaut man sich immer ein möglichst großes $n$ an (also $\lim\limits_{n \rightarrow \infty}{ f(n) }$), so gilt für einen Algorithmus $g$ der anfangs schneller wächst als $f$, dann aber ab $n = 10000$ von f überholt wird immer noch $g \in O( f(n) )$.
\end{frame}


%Groß O
\begin{frame}
  \frametitle{Groß-O: Definition}
Das Groß-O ist mathematisch so definiert:\\
\ \\
\begin{itemize}
	\item $O( f(n) ) = \{ g(n) | \exists c \in \mathbb{R}_+ : \exists  n_0 \in \mathbb{N}_0 : \forall n \geq n_0 : g(n) \leq c \cdot f(n) \}$
\end{itemize}
\ \\
$O( f(n) )$ sind also alle g die ab irgendeinem $n_0$ von f überholt werden.
\end{frame}


%Omega
\begin{frame}
  \frametitle{$\Omega$: Definition}
Das $\Omega$ ist mathematisch so definiert:\\
\ \\
\begin{itemize}
	\item $\Omega( f(n) ) = \{ g(n) | \exists c \in \mathbb{R}_+ : \exists  n_0 \in \mathbb{N}_0 : \forall n \geq n_0 : g(n) \geq c \cdot f(n) \}$
\end{itemize}
\ \\
$\Omega( f(n) )$ sind also alle g die ab einem $n_0$ immer schneller oder gleich schnell wachsen wie f.
\end{frame}


%Theta
%Omega
\begin{frame}
  \frametitle{$\Theta$: Definition}
Das $\Theta$ ist mathematisch so definiert:\\
\ \\
\begin{itemize}
	\item $\Theta( f(n) ) = \{ g(n) | \exists c_1, c_2 \in \mathbb{R}_+ : \exists  n_0 \in \mathbb{N}_0 : \forall n \geq n_0 : c_1 f(n) \leq g(n) \leq c_2 f(n) \}$
	\item Alternativ: $\Theta( f(n) ) = O( f(n) ) \cap \Omega( f(n) )$
\end{itemize}
\ \\
$\Theta( f(n) )$ legt ab $n_0$ einen Bereich um f ab in dem g liegt.
\end{frame}








\end{document}
