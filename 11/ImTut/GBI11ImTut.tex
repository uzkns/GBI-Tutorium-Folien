\documentclass[ngerman,hyperref={pdfpagelabels=false}]{beamer}

% -----------------------------------------------------------------------------

\graphicspath{{images/}}

% -----------------------------------------------------------------------------

\usetheme{KIT}

\setbeamercovered{transparent}
%\setbeamertemplate{enumerate items}[ball]

\newenvironment<>{KITtestblock}[2][]
{\begin{KITcolblock}<#1>{#2}{KITblack15}{KITblack50}}
{\end{KITcolblock}}

\usepackage[ngerman,english]{babel}
\usepackage[utf8]{inputenc}
\usepackage[TS1,T1]{fontenc}
\usepackage{array}
\usepackage{multicol}
\usepackage[absolute,overlay]{textpos}
\usepackage{beamerKITdefs}
\usepackage{amsfonts}

\newcommand{\code}[1]{\texttt{#1}}


\pdfpageattr {/Group << /S /Transparency /I true /CS /DeviceRGB>>}	%required to prevent color shifting withd transparent images


\title{Tutorium 17, \#11}
\subtitle{Max Göckel-- \textit{uzkns@student.kit.edu}}

\author[Max Göckel]{Max Göckel}
\institute{Institut für Theoretische Informatik - Grundbegriffe der Informatik}

\TitleImage[width=\titleimagewd,height=\titleimageht]{titel}

\KITinstitute{Institut f\"ur Theoretische Informatik}
\KITfaculty{Fakult\"at f\"ur Informatik}

% -----------------------------------------------------------------------------

\begin{document}
\setlength\textheight{7cm} %required for correct vertical alignment, if [t] is not used as documentclass parameter


% title frame
\begin{frame}
  \maketitle
\end{frame}


%Landau Notation
\begin{frame}
  \frametitle{Algorithmen: O-Kalkül}
Um Algorithmen zu vergleichen und einzuordnen gibt es verschiedene Schranken: \\
\begin{itemize}

	\item Obere Schranke $O$ (Groß-O): In $O( f(n) )$ sind alle Funktionen die langsamer oder gleich schnell wie f wachsen.
	\begin{itemize}
		\item $\{ g(n) | \exists c \in \mathbb{R}_+ : \exists  n_0 \in \mathbb{N}_0 : \forall n \geq n_0 : g(n) \leq c \cdot f(n) \}$
	\end{itemize}

	\item Untere Schranke $\Omega$ (Omega): In $\Omega ( f(n) )$ sind alle Funktionen die schneller oder gleich schnell wie f wachsen.
	\begin{itemize}
		\item $\{ g(n) | \exists c \in \mathbb{R}_+ : \exists  n_0 \in \mathbb{N}_0 : \forall n \geq n_0 : g(n) \geq c \cdot f(n) \}$
	\end{itemize}

	\item Mittlere Schranke $\Theta$ (Theta): In $\Theta ( f(n) )$ sind alle Funktionen die genau gleich schnell wie f wachsen.
	\begin{itemize}
		\item $\{ g(n) | \exists c_1, c_2 \in \mathbb{R}_+ : \exists  n_0 \in \mathbb{N}_0 : \forall n \geq n_0 : c_1 f(n) \leq g(n) \leq c_2 f(n) \}$
		\item $O( f(n) ) \cap \Omega( f(n) )$
	\end{itemize}

\end{itemize}
\end{frame}


%Beispiel
\begin{frame}
  \frametitle{O-Kalkül: Beispiel}
Sei $f(n) = \Pi \cdot n^{10}$ und $g(n) = \frac{1}{e^9} \cdot n^{10}$.\\
\ \\
Finde passende $c$, $n_0$ um das Laufzeitverhalten von $f$ und $g$ zu vergleichen.
\end{frame}


%Beispiel Lösung
\begin{frame}
  \frametitle{O-Kalkül: Beispiel}
Sei $f(n) = \Pi \cdot n^{10}$ und $g(n) = \frac{1}{e^9} \cdot n^{10}$.\\
\ \\
\begin{itemize}
	\item $f(n) = \Pi n^{10} \leq ( e^9 \Pi ) \cdot \frac{1}{e^9} n^{10} = e^9 \Pi \cdot g(n)$
	\begin{itemize}
		\item $f(n) \in O( g(n) )$ mit $n_0 = 1$ und $c = e^9 \Pi$
	\end{itemize}
	\item $g(n) = \frac{1}{e^9} \cdot n^{10} \leq n^{10} \leq \Pi n^{10} = 1 \cdot f(n)$
	\begin{itemize}
		\item $g(n) \in O( f(n) )$ mit $n_0 = 1$ und $c=1$
	\end{itemize}
\end{itemize}
\ \\
Also ist $g(n) \in \Theta( f(n) )$.
\end{frame}


%Aufgaben
\begin{frame}
  \frametitle{O-Kalkül: Aufgaben}

\begin{itemize}
	\item $f(n) = 5n^2 + 3$ und $g(n) = \frac{1}{2} n^2$
	\item $f(n) = 3^n$ und $g(n) = 5^n$
	\item $f(n) = 3n^7 + 4n^6 - n^3 + n$ und $g(n) = 2n^7 - n^5 + 3n^2$
\end{itemize}
\ \\
Finde passende $c$, $n_0$ um das Laufzeitverhalten von $f$ und $g$ zu vergleichen.
\end{frame}


%Mastertheorem
\begin{frame}
  \frametitle{Mastertheorem}
Das Mastertheorem ermöglicht Laufzeitabschätzungen für rekursive Algorithmen. \\
Die Funktionen der Algorithmen müssen folgende Form besitzen, damit das Mastertheorem angewendet werden kann: \\
$T(n) = a \cdot T(\frac{n}{b}) + f(n)$.
\end{frame}



\begin{frame}
\frametitle{Mastertheorem}
$T(n) = a \cdot T(\frac{n}{b}) + f(n)$. \\
\ \\
\begin{itemize}
	\item $a$: Anzahl der Teilprobleme
	\item $\frac{n}{b}$: Größe der Teilprobleme
	\item $f(n)$: Von $T(n)$ unabhängige Funktion zur Kombination der Teilprobleme
\end{itemize}

Bsp.: $T_1(n) = a \cdot T_1(\frac{n}{10}) + 3n^2$
\end{frame}


%Mastertheorem Fälle
\begin{frame}
  \frametitle{Mastertheorem: Fälle}
Bei der Laufzeit der Form $T(n) = a \cdot T(\frac{n}{b}) + f(n)$ gibt es 3 Fälle:\\
\ \\
\begin{enumerate}
	\item Ist $f(n) \in O( n^{{log}_{b}a - \epsilon} )$ für ein $\epsilon > 0 \Rightarrow T(n) \in \Theta( n^{{log}_{b} a} )$
	\item Ist $f(n) \in \Theta( n^{{log}_b a} ) \Rightarrow T(n) \in \Theta( n^{{log}_b a} \cdot {log}(n) )$
	\item Ist $f(n) \in \Omega( n^{{log}_b a+\epsilon} )$ für ein $\epsilon > 0$ und $\exists d \in (0,1)$, sodass für ein großes $n$ gilt: $a \cdot f(\frac{n}{b}) \leq d \cdot f(n) \Rightarrow T(n) \in \Theta( f(n) )$
\end{enumerate}
\end{frame}


%MT Aufgaben
\begin{frame}
  \frametitle{Mastertheorem: Aufgaben}
\begin{enumerate}
	\item $ T(n) = 2 T( \frac{n}{2} ) + 10 n $
	\item $ 20 n^2 + T( \frac{n}{2} ) \cdot 8 $
	\item $ 4 T( \frac{n}{4} ) + n^2 $
\end{enumerate}
\end{frame}


%LÖSUNG
\begin{frame}
  \frametitle{Mastertheorem: Lösung}
\begin{enumerate}
	\item $a = 2, b = 2, f(n) = 10n$; ${log}_b a = 1$; $f(n) = 10n \in \Theta(n^1) \Rightarrow T(n) \in \Theta(n \cdot log n)$
	\item  $a = 8, b = 2, f(n) = 20n^2$;  ${log}_b a = 3$; $f(n) = 20n^2 \in O( n^{{log}_b a-\epsilon} ) = O( n^{3-\epsilon} )$ für $\epsilon = \frac{1}{2} \Rightarrow T(n) \in \Theta( n^3 )$
\end{enumerate}
\end{frame}


%LÖSUNG
\begin{frame}
  \frametitle{Mastertheorem: Lösung der 3.}
$a = 4, b = 4, f(n) = n^2$, \\
${log}_b a = 1$, \\
$f(n) = n^2 \in \Omega( n^{{log}_b a+\epsilon} ) = \Omega( n^{1+\epsilon} )$, \\
Für $\epsilon = \frac{1}{2}$. Es soll gelten $a f(\frac{n}{b}) \leq d f(n), d \in (0,1)$.\\
\ \\
$a f(\frac{n}{b}) = 4 f(\frac{n}{4}) = \frac{n^2}{4} \leq d n^2 = d f(n)$ mit $d = \frac{1}{4}$. \\
$\Rightarrow T(n) \in \Theta(n^2)$

\end{frame}


%Klausuraufgaben
\begin{frame}
  \frametitle{Klausuraufgaben}
Um der Vorlesung nicht zu weit "voraus" zu sein, machen wir jetzt ein paar Klausuraufgaben zu Themen die wir schon hatten.\\
\end{frame}

























\end{document}
