% !TEX TS-program = pdflatex
% !TEX encoding = UTF-8 Unicode

% This is a simple template for a LaTeX document using the "article" class.
% See "book", "report", "letter" for other types of document.

\documentclass[11pt]{article} % use larger type; default would be 10pt

\usepackage[utf8]{inputenc} % set input encoding (not needed with XeLaTeX)

%%% Examples of Article customizations
% These packages are optional, depending whether you want the features they provide.
% See the LaTeX Companion or other references for full information.

%%% PAGE DIMENSIONS
\usepackage{geometry} % to change the page dimensions
\geometry{a4paper} % or letterpaper (US) or a5paper or....
% \geometry{margin=2in} % for example, change the margins to 2 inches all round
% \geometry{landscape} % set up the page for landscape
%   read geometry.pdf for detailed page layout information

\usepackage{graphicx} % support the \includegraphics command and options

% \usepackage[parfill]{parskip} % Activate to begin paragraphs with an empty line rather than an indent

%%% PACKAGES
\usepackage{booktabs} % for much better looking tables
\usepackage{array} % for better arrays (eg matrices) in maths
\usepackage{paralist} % very flexible & customisable lists (eg. enumerate/itemize, etc.)
\usepackage{verbatim} % adds environment for commenting out blocks of text & for better verbatim
\usepackage{subfig} % make it possible to include more than one captioned figure/table in a single float
% These packages are all incorporated in the memoir class to one degree or another...

%%% HEADERS & FOOTERS
\usepackage{fancyhdr} % This should be set AFTER setting up the page geometry
\pagestyle{fancy} % options: empty , plain , fancy
\renewcommand{\headrulewidth}{0pt} % customise the layout...
\lhead{}\chead{}\rhead{}
\lfoot{}\cfoot{\thepage}\rfoot{}

%%% SECTION TITLE APPEARANCE
\usepackage{sectsty}
\allsectionsfont{\sffamily\mdseries\upshape} % (See the fntguide.pdf for font help)
% (This matches ConTeXt defaults)

%%% ToC (table of contents) APPEARANCE
\usepackage[nottoc,notlof,notlot]{tocbibind} % Put the bibliography in the ToC
\usepackage[titles,subfigure]{tocloft} % Alter the style of the Table of Contents
\usepackage[ngerman,english]{babel}
\usepackage[utf8]{inputenc}
\usepackage[TS1,T1]{fontenc}
\usepackage{array}
\usepackage{multicol}
\usepackage[absolute,overlay]{textpos}
\usepackage{amsfonts}
\usepackage{amsmath}
\usepackage{tabto}
\renewcommand{\cftsecfont}{\rmfamily\mdseries\upshape}
\renewcommand{\cftsecpagefont}{\rmfamily\mdseries\upshape} % No bold!

%%% END Article customizations

%%% The "real" document content comes below...

\title{Altklausuraufgaben GBI}
\author{Max Göckel, Tutorium 42}
\date{} % Activate to display a given date or no date (if empty),
         % otherwise the current date is printed 

\begin{document}
\maketitle

\section{WS 16/17}



\subsection{Aufgabe 2.}
a) Gelten die Klammereinsparungsregeln der Aussagenlogik auch für die Prädikatenlogik? \\
- \quad Ja \quad | \quad Nein \\
\ \\
b) Listen sie alle in der VL eingeführten  Aussagenlogik-Konnektive auf:\\
\ \\
\ \\
e) Geben Sie die Anzahl der Knoten und Kanten an die der Graph
$
\begin{pmatrix}
1 & 0 & 1 & 0 \\
1 & 0 & 0 & 1 \\
0 & 1 & 0 & 1 \\
0 & 1 & 1 & 0
\end{pmatrix}
$ hat.\\
\ \\
\ \\
\ \\
f) Wie viele Wurzeln hat ein gerichteter Baum?
\ \\
\ \\
\subsection{Aufgabe 3.}
Die Fibonacci-Zahlen sind wie folgt induktiv definiert:\\
\begin{center}
$F_0 = 0$ \\
$F_1 = 1$ \\
und $\forall n \geq 2 \in \mathbb{N}_0: $ \\
$F_n = F_{n-1} + F_{n-2}$
\end{center}
\ \\
a) Beweisen sie folgende Aussage durch vollständige Induktion: \\
$\forall n \in \mathbb{N}_+: \sum_{i=1}^n F_i = F_{n+2} - 1$ \\
\ \\
b) Beweisen sie folgende Aussage durch vollständige Induktion: \\
$\forall n \in \mathbb{N}_+: \sum_{i=1}^n F_{i}^2 = F_n F_{n+1}$
\ \\
\ \\

\subsection{Aufgabe 4.}
a) + b) Geben sie zum Graph $G_1$ an der Tafel die Adjazenzmatrix und die Wegematrix an: \\
\ \\
\ \\
\ \\
\ \\
\ \\
\ \\
\ \\
c) Ist $G_1$ azyklisch? \quad Ja \quad | \quad Nein \\
\ \\
d) Ist $G_1$ streng zusammenhängend? \quad Ja \quad | \quad Nein \\
\ \\
e) Geben sie den maximalen eingangs- und Ausgangsgrad der Knoten in $G_1$ an:
\ \\
\ \\
\ \\
\section{SS 17}
\subsection{Aufgabe 2.}
f) welche der folgenden Mengen sind Alphabete? \\
\begin{itemize}
	\item $\{ 0, 1, 0, 1 \}$
	\item $\{ x \in \mathbb{N} | x^2 \geq 1000 \}$
	\item $\{ x \in \mathbb{N} | 3 \cdot x + 100 \leq 1000 \}$
	\item $\emptyset$
	\item $\{ 1,2,3 \} x \{ a,b,c \}$
	\item $\mathbb{N}_+$
\end{itemize}
\ \\
h) Welche Eigenschaften muss eine Relation haben, damit sie eine Äquivalenzrelation ist? \\
\ \\
\ \\
\ \\

























\end{document}
