% !TEX TS-program = pdflatex
% !TEX encoding = UTF-8 Unicode

% This is a simple template for a LaTeX document using the "article" class.
% See "book", "report", "letter" for other types of document.

\documentclass[11pt]{article} % use larger type; default would be 10pt

\usepackage[utf8]{inputenc} % set input encoding (not needed with XeLaTeX)

%%% Examples of Article customizations
% These packages are optional, depending whether you want the features they provide.
% See the LaTeX Companion or other references for full information.

%%% PAGE DIMENSIONS
\usepackage{geometry} % to change the page dimensions
\geometry{a4paper} % or letterpaper (US) or a5paper or....
% \geometry{margin=2in} % for example, change the margins to 2 inches all round
% \geometry{landscape} % set up the page for landscape
%   read geometry.pdf for detailed page layout information

\usepackage{graphicx} % support the \includegraphics command and options

% \usepackage[parfill]{parskip} % Activate to begin paragraphs with an empty line rather than an indent

%%% PACKAGES
\usepackage{booktabs} % for much better looking tables
\usepackage{array} % for better arrays (eg matrices) in maths
\usepackage{paralist} % very flexible & customisable lists (eg. enumerate/itemize, etc.)
\usepackage{verbatim} % adds environment for commenting out blocks of text & for better verbatim
\usepackage{subfig} % make it possible to include more than one captioned figure/table in a single float
% These packages are all incorporated in the memoir class to one degree or another...

%%% HEADERS & FOOTERS
\usepackage{fancyhdr} % This should be set AFTER setting up the page geometry
\pagestyle{fancy} % options: empty , plain , fancy
\renewcommand{\headrulewidth}{0pt} % customise the layout...
\lhead{}\chead{}\rhead{}
\lfoot{}\cfoot{\thepage}\rfoot{}

%%% SECTION TITLE APPEARANCE
\usepackage{sectsty}
\allsectionsfont{\sffamily\mdseries\upshape} % (See the fntguide.pdf for font help)
% (This matches ConTeXt defaults)

%%% ToC (table of contents) APPEARANCE
\usepackage[nottoc,notlof,notlot]{tocbibind} % Put the bibliography in the ToC
\usepackage[titles,subfigure]{tocloft} % Alter the style of the Table of Contents
\usepackage{amsfonts}
\usepackage{extarrows}
\renewcommand{\cftsecfont}{\rmfamily\mdseries\upshape}
\renewcommand{\cftsecpagefont}{\rmfamily\mdseries\upshape} % No bold!

%%% END Article customizations

%%% The "real" document content comes below...

\title{Aufgaben und Lösungen zum O-Kalkül}
\author{Max Göckel, Tutorium 17}
\date{} % Activate to display a given date or no date (if empty),
         % otherwise the current date is printed 

\begin{document}
\maketitle


\section{Aufgabe 1}

Vergleiche das Laufzeitverhalten von $f(n) = 5n^2 + 3$ und $g(n) = \frac{1}{2} n^2$.\\
Finde passende $c$, $n_0$ und begründe mittels Umformungen und Abschätzungen.

\subsection{Lösung}

$f(n) = 5n^2 + 3 \leq 5n^2 + 3n^2 = 8n^2 = 16(\frac{1}{2} n^2) = 16 \cdot g(n)$.\\
\ \\
Damit ist $c = 16$ und somit $n_0 = 1$.


\section{Aufgabe 2}

Vergleiche das Laufzeitverhalten von $f(n) = 3^n$ und $g(n) = 5^n$\\
Finde passende $c$, $n_0$ und begründe mittels Umformungen und Abschätzungen.


\subsection{Lösung}

$f(n) \leq c \cdot g(n)$ gilt trivialerweise für alle $c \in \mathbb{N}$.\\
\ \\
Für $g(n) \leq  c \cdot f(n)$ muss $\frac{ g(n) }{ f(n) } \leq c$ für ein beliebiges und festes $c \in \mathbb{R}_+$ gelten.\\

$\frac{ g(n) }{ f(n) } = \frac{ 5^n }{ 3^n } = \frac{5}{3}^n \xrightarrow{\text{$n \rightarrow \infty$}} \infty $\\
Damit gibt es kein passendes $c \in \mathbb{R}_+$.\\
\ \\
Damit ist $f(n) \in O( g(n) )$ und auch $g(n) \in \Omega( f(n) )$, aber nicht $f(n) \in O( g(n) )$.


\section{Aufgabe 3}

Vergleiche das Laufzeitverhalten von $f(n) = 3n^7 + 4n^6 - n^3 + n$ und $g(n) = 2n^7 - n^5 + 3n^2$\\
Finde passende $c$, $n_0$ und begründe mittels Umformungen und Abschätzungen.


\subsection{Lösung}

\begin{align}
  f(n)
  &= 3n^7 + 4n^6 - n^3 + n
  \notag\\ &\leq 3n^7 + 4n^6 + n
  \notag\\ &\leq 3n^7 + 4n^7 + n^7
  \notag\\ &= 8n^7
  \notag\\ &\leq 8n^7 + 24n^2
  \notag\\ &= 8(2n^7 - n^7 + 3n^2)
  \notag\\ &\leq 8(2n^7 - n^5 + 3n^2)
  \notag\\ &= 8 \cdot g(n)
\end{align}
für $n_0 \geq 1$, also gilt $f(n) \in O( g(n) )$.


\begin{align}
  g(n)
  &= 2n^7 - n^5 + 3n^2
  \notag\\ &\leq 2n^7 + 3n^2
  \notag\\ &\leq 32n^7 + 4n^6 + n
  \notag\\ &= 3n^7 - n^7 + 4n^6 + n
  \notag\\ &\leq 3n^7 - n^3 + 4n^6 + n
  \notag\\ &= 1 \cdot f(n)
\end{align}
für $c = n_0 = 1$ also gilt $g(n) \in O( f(n) )$.\\
\ \\
Mit $(1)$ und $(2)$ gilt damit auch $g(n) \in \Theta( f(n) )$.










\end{document}
