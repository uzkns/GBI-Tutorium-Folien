\documentclass[ngerman,hyperref={pdfpagelabels=false}]{beamer}

% -----------------------------------------------------------------------------

\graphicspath{{images/}}

% -----------------------------------------------------------------------------

\usetheme{KIT}

\setbeamercovered{transparent}
%\setbeamertemplate{enumerate items}[ball]

\newenvironment<>{KITtestblock}[2][]
{\begin{KITcolblock}<#1>{#2}{KITblack15}{KITblack50}}
{\end{KITcolblock}}

\usepackage[ngerman,english]{babel}
\usepackage[utf8]{inputenc}
\usepackage[TS1,T1]{fontenc}
\usepackage{array}
\usepackage{multicol}
\usepackage[absolute,overlay]{textpos}
\usepackage{beamerKITdefs}
\usepackage{amsfonts}


\pdfpageattr {/Group << /S /Transparency /I true /CS /DeviceRGB>>}	%required to prevent color shifting withd transparent images


\title{Tutorium 17, \#6}
\subtitle{Max Göckel-- \textit{uzkns@student.kit.edu}}

\author[Max Göckel]{Max Göckel}
\institute{Institut für Theoretische Informatik - Grundbegriffe der Informatik}

\TitleImage[width=\titleimagewd,height=\titleimageht]{titel}

\KITinstitute{Institut f\"ur Theoretische Informatik}
\KITfaculty{Fakult\"at f\"ur Informatik}

% -----------------------------------------------------------------------------

\begin{document}
\setlength\textheight{7cm} %required for correct vertical alignment, if [t] is not used as documentclass parameter


% title frame
\begin{frame}

\maketitle
\end{frame}


%Aufgabe%
\begin{frame}
\frametitle{Die MIMA: Aufgabe}
Beschreibt die folgenden Befehle.\\
\ \\
JMP a\\
LDV a\\
HALT\\
ADD a\\
\end{frame}


%LÖSUNG%
\begin{frame}
\frametitle{Die MIMA: Aufgabe}
Beschreibt die folgenden Befehle.\\
\ \\
JMP a: Springt zum Befehl an Stelle a (oder in GBI: Mit Markierung a)\\
LDV a: Lädt den Wert an Adresse a in den Akku\\
HALT: Stoppt die MIMA\\
ADD a: Addiert den Wert an Adresse a auf den Akku drauf (Ergebnis landet wieder im Akku)\\
\end{frame}

\begin{frame}
\frametitle{Wiederholung}
Wir kennen schon: \\
\begin{itemize}
	\item $x-y$
	\item $x$ div $2$
\end{itemize}

\end{frame}

\begin{frame}
\frametitle{Erinnerung: div}
\begin{itemize}
\item adr1 $=$ adr1 div $2$:\\
\ \\
\begin{tabular}{lll}
	&LDV& EINS\\
	&NOT&\\
	&AND& adr1\\
	&RAR&\\
	&STV& adr1\\
	&HALT&\\
\end{tabular}\\
\end{itemize}
\end{frame}

\begin{frame}
\frametitle{Erinnerung: Minus}
\begin{itemize}
\item adr3 $=$ adr1 $-$ adr2:\\
\ \\
\begin{tabular}{lll}
&LDV&adr2\\
&NOT&\\
&ADD&EINS\\
&ADD&adr1\\
&STV&adr3\\
&HALT&\\
\end{tabular}
\end{itemize}
\end{frame}



\begin{frame}
\frametitle{Aufgabe}
Was macht folgendes Programm?\\
\begin{minipage}{0.45\textwidth}
	\scalebox{0.65}{
		\begin{tabular}{r|lll}
			Zeilen & Code &  & \\
			\hline
			1 && LDC & 4 \\
			2 && STV & adr0 \\
			3 && ADD & EINS \\
			4 && STV & adr1\\
			5 && ADD & EINS\\
			6 && STV & adr2 \\
			7 && LDC & 0\\
			8 && STIV & adr0\\
			9 && ADD & EINS\\
			10 && STIV & adr1\\
			11 &$m1:$& LDIV & adr1\\
			12 && STV & adr3\\
			13 && LDIV & adr0\\
			14 && ADD & adr3\\
			15 && STIV & adr2\\
			16 && LDV & adr0\\
			17 && ADD & EINS\\
			18 && STV & adr0\\
			19 && ADD & EINS\\
			20 && STV & adr1\\
			21 && ADD & EINS\\
			22 && STV & adr2\\
			23 && JMP & m1\\
			
		\end{tabular}
	}
\end{minipage}
\begin{minipage}{0.50\textwidth}
	\scalebox{0.65}{Der Speicher hat folgende Form:}
	\ \\
	\ \\
	\scalebox{0.65}{
		\begin{tabular}{l|l}
			Adresse & Wert\\ 
			\hline
			0 aka adr0 & ? \\
			1 aka adr1 & ? \\
			2 aka adr2 & ? \\
			3 aka adr3 & ? \\
			4 & ? \\
			5 & ? \\
			6 & ? \\
			\vdots & \vdots\\
		\end{tabular}
	}\\
\end{minipage}
\end{frame}


\begin{frame}
\frametitle{Aufgabe}

Schreibe ein Programm dass $z = x$ mod $y$ ausführt.\\
Dabei:
\begin{itemize}
	\item x = Wert an Adresse adr1
	\item y = Wert an Adresse adr2
	\item z = Wert an Adresse adr3
	\item x aus $\mathbb{N}_0$
	\item y aus $\mathbb{N}_+$
\end{itemize}
\end{frame}

\begin{frame}
\frametitle{Lösung}

Schreibe ein Programm dass $z = x$ mod $y$ ausführt.\\
\ \\
\qquad LDV adr1 \\
\qquad STV adr3 \\
m1: LDV adr2 \\
\qquad NOT \\
\qquad ADD EINS \\
\qquad ADD adr3 \\
\qquad JMN m0 \\
\qquad STV adr3 \\
\qquad JMP m1 \\
m0: HALT

\end{frame}



\begin{frame}
\frametitle{Aufgabe}

Schreibe ein Programm dass $z = x$ div $y$ ausführt.\\
Dabei:
\begin{itemize}
	\item x = Wert an Adresse adr1
	\item y = Wert an Adresse adr2
	\item z = Wert an Adresse adr3
	\item x aus $\mathbb{N}_0$
	\item y aus $\mathbb{N}_+$
\end{itemize}

\end{frame}


\begin{frame}
\frametitle{Aufgabe}

Schreibe ein Programm dass $z = x$ div $y$ ausführt.\\
{\small \qquad LDC 0 \\
	\qquad STV counter \\
	\qquad LDV adr1 \\
	\qquad STV rest \\
	m1: LDV adr2 \\
	\qquad NOT \\
	\qquad ADD EINS \\
	\qquad ADD rest \\
	\qquad JMN m2 \\
	\qquad STV rest \\
	\qquad LDV counter \\
	\qquad ADD EINS \\
	\qquad STV counter \\
	\qquad JMP m1 \\
	m2: LDV counter \\
	\qquad STV adr3 \\
	\qquad HALT \\
}
\end{frame}






%Aufgabe
\begin{frame}
\frametitle{Die MIMA: Aufgabe}
An Speicherstelle adr1 steht ein Wert $x \in \mathbb{N}_+$.\\
\ \\
Schreibe ein Programm, dass den Wert an adr2 $x$-mal multipliziert (adr2 $\cdot$ x)
\end{frame}


%LÖSUNG
\begin{frame}
\frametitle{Die MIMA: Aufgabe}
Schreibe ein Programm, dass den Wert an adr2 $x$-mal multipliziert:\\
\qquad LDC 0 \\
\qquad NOT \\
\qquad STV adr3 \textit{;-1 an adr3} \\
\qquad LDV adr2 \textit{;Wert von adr2 abspeichern} \\
\qquad STV adr4 \\
m1: LDV adr1 \\
\qquad ADD adr3 \\
\qquad STV adr1 \textit{;adr1 - 1 rechnen und speichern} \\
\qquad LDV adr2 \\
\qquad ADD adr4 \\
\qquad STV adr2 \textit{;adr2 + adr4 rechnen und speichern} \\
\qquad LDV adr3 \\
\qquad JMN m2 \\
\qquad JMP m1 \\
m2: HALT
\end{frame}


\begin{frame}
\frametitle{Die MIMA: Befehle}
LDC c - lädt Konstante c in den Akku \\
LDV a - Lädt Wert an Stelle a in den Akku \\
STV a - speicher Wert vom Akku in Stelle a \\
ADD a - Addiert Wert an Stelle a auf den Wert im Akku drauf (Ergebn. $\rightarrow$ Akku) \\
AND a - VerUNDet Wert v.St. a und Akku $\rightarrow$ Akku \\
OR a - VerODERt Wert v.St. a und Akku $\rightarrow$ Akku \\
XOR a - VerXORt Wert v.St. a und Akku $\rightarrow$ Akku \\
EQL a - Falls Akku = Wert a.St. a $\rightarrow$ Akku -1 in Akku, sonst 0 \\
JMP a - Springt zum Befehl mit Marker a \\
JMN a - Springt zum Befehl a wenn Akku < 0 \\
HALT - Endet die Auführung der MIMA \\
NOT - Invertiert den Akku (0 $\rightarrow$ 1, 1 $\rightarrow$ 0) \\
RAR - Rotiert den Akku um eins nach rechts\\
\ \\
Der Wert des Akku wird immer überschreiben wenn man neue Werte lädt, also Werte die man mehrfach braucht zwischenspeichern (STV).
\end{frame}




\end{document}
